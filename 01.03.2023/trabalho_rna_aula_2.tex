% Options for packages loaded elsewhere
\PassOptionsToPackage{unicode}{hyperref}
\PassOptionsToPackage{hyphens}{url}
%
\documentclass[
]{article}
\usepackage{amsmath,amssymb}
\usepackage{lmodern}
\usepackage{iftex}
\ifPDFTeX
  \usepackage[T1]{fontenc}
  \usepackage[utf8]{inputenc}
  \usepackage{textcomp} % provide euro and other symbols
\else % if luatex or xetex
  \usepackage{unicode-math}
  \defaultfontfeatures{Scale=MatchLowercase}
  \defaultfontfeatures[\rmfamily]{Ligatures=TeX,Scale=1}
\fi
% Use upquote if available, for straight quotes in verbatim environments
\IfFileExists{upquote.sty}{\usepackage{upquote}}{}
\IfFileExists{microtype.sty}{% use microtype if available
  \usepackage[]{microtype}
  \UseMicrotypeSet[protrusion]{basicmath} % disable protrusion for tt fonts
}{}
\makeatletter
\@ifundefined{KOMAClassName}{% if non-KOMA class
  \IfFileExists{parskip.sty}{%
    \usepackage{parskip}
  }{% else
    \setlength{\parindent}{0pt}
    \setlength{\parskip}{6pt plus 2pt minus 1pt}}
}{% if KOMA class
  \KOMAoptions{parskip=half}}
\makeatother
\usepackage{xcolor}
\usepackage[margin=1in]{geometry}
\usepackage{color}
\usepackage{fancyvrb}
\newcommand{\VerbBar}{|}
\newcommand{\VERB}{\Verb[commandchars=\\\{\}]}
\DefineVerbatimEnvironment{Highlighting}{Verbatim}{commandchars=\\\{\}}
% Add ',fontsize=\small' for more characters per line
\usepackage{framed}
\definecolor{shadecolor}{RGB}{248,248,248}
\newenvironment{Shaded}{\begin{snugshade}}{\end{snugshade}}
\newcommand{\AlertTok}[1]{\textcolor[rgb]{0.94,0.16,0.16}{#1}}
\newcommand{\AnnotationTok}[1]{\textcolor[rgb]{0.56,0.35,0.01}{\textbf{\textit{#1}}}}
\newcommand{\AttributeTok}[1]{\textcolor[rgb]{0.77,0.63,0.00}{#1}}
\newcommand{\BaseNTok}[1]{\textcolor[rgb]{0.00,0.00,0.81}{#1}}
\newcommand{\BuiltInTok}[1]{#1}
\newcommand{\CharTok}[1]{\textcolor[rgb]{0.31,0.60,0.02}{#1}}
\newcommand{\CommentTok}[1]{\textcolor[rgb]{0.56,0.35,0.01}{\textit{#1}}}
\newcommand{\CommentVarTok}[1]{\textcolor[rgb]{0.56,0.35,0.01}{\textbf{\textit{#1}}}}
\newcommand{\ConstantTok}[1]{\textcolor[rgb]{0.00,0.00,0.00}{#1}}
\newcommand{\ControlFlowTok}[1]{\textcolor[rgb]{0.13,0.29,0.53}{\textbf{#1}}}
\newcommand{\DataTypeTok}[1]{\textcolor[rgb]{0.13,0.29,0.53}{#1}}
\newcommand{\DecValTok}[1]{\textcolor[rgb]{0.00,0.00,0.81}{#1}}
\newcommand{\DocumentationTok}[1]{\textcolor[rgb]{0.56,0.35,0.01}{\textbf{\textit{#1}}}}
\newcommand{\ErrorTok}[1]{\textcolor[rgb]{0.64,0.00,0.00}{\textbf{#1}}}
\newcommand{\ExtensionTok}[1]{#1}
\newcommand{\FloatTok}[1]{\textcolor[rgb]{0.00,0.00,0.81}{#1}}
\newcommand{\FunctionTok}[1]{\textcolor[rgb]{0.00,0.00,0.00}{#1}}
\newcommand{\ImportTok}[1]{#1}
\newcommand{\InformationTok}[1]{\textcolor[rgb]{0.56,0.35,0.01}{\textbf{\textit{#1}}}}
\newcommand{\KeywordTok}[1]{\textcolor[rgb]{0.13,0.29,0.53}{\textbf{#1}}}
\newcommand{\NormalTok}[1]{#1}
\newcommand{\OperatorTok}[1]{\textcolor[rgb]{0.81,0.36,0.00}{\textbf{#1}}}
\newcommand{\OtherTok}[1]{\textcolor[rgb]{0.56,0.35,0.01}{#1}}
\newcommand{\PreprocessorTok}[1]{\textcolor[rgb]{0.56,0.35,0.01}{\textit{#1}}}
\newcommand{\RegionMarkerTok}[1]{#1}
\newcommand{\SpecialCharTok}[1]{\textcolor[rgb]{0.00,0.00,0.00}{#1}}
\newcommand{\SpecialStringTok}[1]{\textcolor[rgb]{0.31,0.60,0.02}{#1}}
\newcommand{\StringTok}[1]{\textcolor[rgb]{0.31,0.60,0.02}{#1}}
\newcommand{\VariableTok}[1]{\textcolor[rgb]{0.00,0.00,0.00}{#1}}
\newcommand{\VerbatimStringTok}[1]{\textcolor[rgb]{0.31,0.60,0.02}{#1}}
\newcommand{\WarningTok}[1]{\textcolor[rgb]{0.56,0.35,0.01}{\textbf{\textit{#1}}}}
\usepackage{graphicx}
\makeatletter
\def\maxwidth{\ifdim\Gin@nat@width>\linewidth\linewidth\else\Gin@nat@width\fi}
\def\maxheight{\ifdim\Gin@nat@height>\textheight\textheight\else\Gin@nat@height\fi}
\makeatother
% Scale images if necessary, so that they will not overflow the page
% margins by default, and it is still possible to overwrite the defaults
% using explicit options in \includegraphics[width, height, ...]{}
\setkeys{Gin}{width=\maxwidth,height=\maxheight,keepaspectratio}
% Set default figure placement to htbp
\makeatletter
\def\fps@figure{htbp}
\makeatother
\setlength{\emergencystretch}{3em} % prevent overfull lines
\providecommand{\tightlist}{%
  \setlength{\itemsep}{0pt}\setlength{\parskip}{0pt}}
\setcounter{secnumdepth}{-\maxdimen} % remove section numbering
\ifLuaTeX
  \usepackage{selnolig}  % disable illegal ligatures
\fi
\IfFileExists{bookmark.sty}{\usepackage{bookmark}}{\usepackage{hyperref}}
\IfFileExists{xurl.sty}{\usepackage{xurl}}{} % add URL line breaks if available
\urlstyle{same} % disable monospaced font for URLs
\hypersetup{
  pdftitle={Trabalho RNA da Aula 2 - 01.03.2023},
  pdfauthor={Afonso Cesar Lelis Brandão - matrícula e e-mail: 72307390@mackenzista.com.br},
  hidelinks,
  pdfcreator={LaTeX via pandoc}}

\title{Trabalho RNA da Aula 2 - 01.03.2023}
\author{Afonso Cesar Lelis Brandão - matrícula e e-mail:
\href{mailto:72307390@mackenzista.com.br}{\nolinkurl{72307390@mackenzista.com.br}}}
\date{2023-03-07}

\begin{document}
\maketitle

\hypertarget{primeira-parte}{%
\section{Primeira Parte}\label{primeira-parte}}

\hypertarget{primeira-questuxe3o}{%
\subsection{Primeira questão}\label{primeira-questuxe3o}}

\begin{enumerate}
\def\labelenumi{\arabic{enumi})}
\tightlist
\item
  Aumente o número máximo de épocas (por exemplo, para 300) e apresente
  o gráfico de erro por época, como também o gráfico de sobreposição de
  saídas (desejada e estimada). Faça uma análise se há melhoras em
  relação aos valores anteriormente apresentados. Verifique o valor
  final de erro e veja se ele consegue chegar a 0 (zero).
\end{enumerate}

\begin{itemize}
\tightlist
\item
  Regressão modelo
\end{itemize}

\begin{Shaded}
\begin{Highlighting}[]
\CommentTok{\# Definindo o tamanho do vetor}
\NormalTok{N}\OtherTok{\textless{}{-}}\DecValTok{10}

\CommentTok{\# Criando o vetor X com valores de 1 a N}
\NormalTok{X}\OtherTok{\textless{}{-}}\FunctionTok{seq}\NormalTok{(}\DecValTok{1}\NormalTok{,N)}

\CommentTok{\# Criando o vetor Y com a função 2X + 3}
\NormalTok{Y}\OtherTok{\textless{}{-}}\DecValTok{2}\SpecialCharTok{*}\NormalTok{X}\SpecialCharTok{+}\DecValTok{3}

\CommentTok{\# Plotando os valores de X e Y em um gráfico de pontos}
\FunctionTok{plot}\NormalTok{(X,Y,}\AttributeTok{type=}\StringTok{"o"}\NormalTok{,}\AttributeTok{xlim=}\FunctionTok{c}\NormalTok{(}\DecValTok{0}\NormalTok{, }\DecValTok{10}\NormalTok{),}\AttributeTok{ylim=}\FunctionTok{c}\NormalTok{(}\DecValTok{0}\NormalTok{,}\DecValTok{25}\NormalTok{))}
\end{Highlighting}
\end{Shaded}

\includegraphics{trabalho_rna_aula_2_files/figure-latex/unnamed-chunk-1-1.pdf}
- Ajustes

\begin{Shaded}
\begin{Highlighting}[]
\CommentTok{\# Normalizando os vetores entre 0 e 1 e transpondo}
\NormalTok{X}\OtherTok{\textless{}{-}}\NormalTok{X}\SpecialCharTok{/}\FunctionTok{max}\NormalTok{(X)}
\NormalTok{Y}\OtherTok{\textless{}{-}}\NormalTok{Y}\SpecialCharTok{/}\FunctionTok{max}\NormalTok{(Y)}
\NormalTok{X}\OtherTok{\textless{}{-}}\FunctionTok{t}\NormalTok{(}\FunctionTok{t}\NormalTok{(X))}

\CommentTok{\# Adicionando um vetor de bias à matriz X}
\NormalTok{bias}\OtherTok{\textless{}{-}}\DecValTok{1}
\NormalTok{X}\OtherTok{\textless{}{-}}\FunctionTok{cbind}\NormalTok{(X,bias)}

\CommentTok{\# Imprimindo as matrizes X e Y}
\FunctionTok{print}\NormalTok{(X)}
\end{Highlighting}
\end{Shaded}

\begin{verbatim}
##           bias
##  [1,] 0.1    1
##  [2,] 0.2    1
##  [3,] 0.3    1
##  [4,] 0.4    1
##  [5,] 0.5    1
##  [6,] 0.6    1
##  [7,] 0.7    1
##  [8,] 0.8    1
##  [9,] 0.9    1
## [10,] 1.0    1
\end{verbatim}

\begin{Shaded}
\begin{Highlighting}[]
\FunctionTok{print}\NormalTok{(}\FunctionTok{cbind}\NormalTok{(Y))}
\end{Highlighting}
\end{Shaded}

\begin{verbatim}
##               Y
##  [1,] 0.2173913
##  [2,] 0.3043478
##  [3,] 0.3913043
##  [4,] 0.4782609
##  [5,] 0.5652174
##  [6,] 0.6521739
##  [7,] 0.7391304
##  [8,] 0.8260870
##  [9,] 0.9130435
## [10,] 1.0000000
\end{verbatim}

\begin{itemize}
\tightlist
\item
  Definindo variáveis e treinando
\end{itemize}

\begin{Shaded}
\begin{Highlighting}[]
\CommentTok{\# Definindo o número máximo de épocas e a taxa de aprendizado}
\NormalTok{epoca\_max}\OtherTok{\textless{}{-}}\DecValTok{300}
\NormalTok{eta}\OtherTok{\textless{}{-}}\FloatTok{0.01}

\CommentTok{\# Inicializando o vetor de pesos W}
\NormalTok{W}\OtherTok{\textless{}{-}}\FunctionTok{c}\NormalTok{(}\DecValTok{0}\NormalTok{,}\DecValTok{1}\NormalTok{)}

\CommentTok{\# Inicializando vetores para armazenar os erros das iterações e das épocas}
\NormalTok{err\_iter}\OtherTok{\textless{}{-}}\FunctionTok{rep}\NormalTok{(}\DecValTok{0}\NormalTok{,N)}
\NormalTok{err\_epoc}\OtherTok{\textless{}{-}}\FunctionTok{rep}\NormalTok{(}\DecValTok{0}\NormalTok{,epoca\_max)}

\CommentTok{\# Iniciando o loop de treinamento}
\ControlFlowTok{for}\NormalTok{ (epoca }\ControlFlowTok{in} \DecValTok{1}\SpecialCharTok{:}\NormalTok{epoca\_max) \{}
  \ControlFlowTok{for}\NormalTok{ (i }\ControlFlowTok{in} \DecValTok{1}\SpecialCharTok{:}\NormalTok{N) \{}
    \CommentTok{\# Calculando a saída estimada para a entrada atual}
\NormalTok{    v }\OtherTok{\textless{}{-}} \FunctionTok{sum}\NormalTok{(X[i,]}\SpecialCharTok{*}\NormalTok{W)}
    \CommentTok{\# Calculando o erro entre a saída desejada e a saída estimada}
\NormalTok{    erro }\OtherTok{\textless{}{-}}\NormalTok{ Y[i] }\SpecialCharTok{{-}}\NormalTok{ v}
    \CommentTok{\# Calculando o delta para atualizar os pesos}
\NormalTok{    delta }\OtherTok{\textless{}{-}}\NormalTok{ eta}\SpecialCharTok{*}\NormalTok{erro}\SpecialCharTok{*}\NormalTok{X[i,]}
    \CommentTok{\# Atualizando os pesos}
\NormalTok{    W }\OtherTok{\textless{}{-}}\NormalTok{ W }\SpecialCharTok{+}\NormalTok{ delta}
    \CommentTok{\# Armazenando o erro da iteração atual}
\NormalTok{    err\_iter[i] }\OtherTok{\textless{}{-}} \FloatTok{0.5}\SpecialCharTok{*}\NormalTok{(erro}\SpecialCharTok{\^{}}\DecValTok{2}\NormalTok{)}
\NormalTok{  \}}
  \CommentTok{\# Armazenando o erro da época atual}
\NormalTok{  err\_epoc[epoca] }\OtherTok{\textless{}{-}} \FunctionTok{sum}\NormalTok{(err\_iter)}
\NormalTok{\}}

\CommentTok{\# Plotando o erro quadrático médio ao longo das épocas}
\FunctionTok{plot}\NormalTok{(err\_epoc, }\AttributeTok{type =} \StringTok{"line"}\NormalTok{, }\AttributeTok{xlab =} \StringTok{"época"}\NormalTok{, }\AttributeTok{ylab =} \StringTok{"erro quadrático médio"}\NormalTok{)}
\end{Highlighting}
\end{Shaded}

\begin{verbatim}
## Warning in plot.xy(xy, type, ...): plot type 'line' will be truncated to first
## character
\end{verbatim}

\includegraphics{trabalho_rna_aula_2_files/figure-latex/unnamed-chunk-3-1.pdf}
- Pesos finais

\begin{Shaded}
\begin{Highlighting}[]
\FunctionTok{print}\NormalTok{(W)}
\end{Highlighting}
\end{Shaded}

\begin{verbatim}
##                bias 
## 0.7146150 0.2231222
\end{verbatim}

\begin{itemize}
\tightlist
\item
  Calculando a saída estimada para todas as entradas e Plotando as
  saídas desejadas e estimadas em um gráfico
\end{itemize}

\begin{Shaded}
\begin{Highlighting}[]
\NormalTok{Y\_estimado}\OtherTok{\textless{}{-}}\NormalTok{X[,}\DecValTok{1}\NormalTok{]}\SpecialCharTok{*}\NormalTok{W[}\DecValTok{1}\NormalTok{]}\SpecialCharTok{+}\NormalTok{W[}\DecValTok{2}\NormalTok{]}

\FunctionTok{plot}\NormalTok{(X[,}\DecValTok{1}\NormalTok{],Y,}\AttributeTok{type =}\StringTok{"n"}\NormalTok{)}
\FunctionTok{points}\NormalTok{(X[,}\DecValTok{1}\NormalTok{],Y,}\AttributeTok{pch=}\DecValTok{1}\NormalTok{,}\AttributeTok{type=}\StringTok{"o"}\NormalTok{,}\AttributeTok{col=}\StringTok{"1"}\NormalTok{)}
\FunctionTok{points}\NormalTok{(X[,}\DecValTok{1}\NormalTok{],Y\_estimado,}\AttributeTok{pch=}\DecValTok{16}\NormalTok{,}\AttributeTok{type=}\StringTok{"p"}\NormalTok{,}\AttributeTok{col=}\StringTok{"2"}\NormalTok{)}
\FunctionTok{legend}\NormalTok{(}\FloatTok{0.1}\NormalTok{,}\FloatTok{0.9}\NormalTok{,}\AttributeTok{legend=}\FunctionTok{c}\NormalTok{(}\StringTok{"Y\_desejado"}\NormalTok{,}\StringTok{"Y\_estimado"}\NormalTok{),}\AttributeTok{col=}\FunctionTok{c}\NormalTok{(}\DecValTok{1}\NormalTok{,}\DecValTok{2}\NormalTok{),}\AttributeTok{pch=}\FunctionTok{c}\NormalTok{(}\DecValTok{1}\NormalTok{,}\DecValTok{16}\NormalTok{))}
\end{Highlighting}
\end{Shaded}

\includegraphics{trabalho_rna_aula_2_files/figure-latex/unnamed-chunk-5-1.pdf}

\begin{Shaded}
\begin{Highlighting}[]
\FunctionTok{print}\NormalTok{(}\FunctionTok{paste}\NormalTok{(}\StringTok{"Erro final do treinamento"}\NormalTok{, err\_epoc[epoca\_max]))}
\end{Highlighting}
\end{Shaded}

\begin{verbatim}
## [1] "Erro final do treinamento 0.0103700261799258"
\end{verbatim}

\hypertarget{resposta}{%
\subsubsection{Resposta}\label{resposta}}

Considerando o contexto abordado, foi possível observar um resultado
significativo ao aumentar o número máximo de épocas para 300 no
treinamento da rede neural. A partir do gráfico de erro por época,
verificou-se que o erro inicialmente elevado apresentou uma rápida
redução nas primeiras iterações, mas, posteriormente, a redução foi
gradual, culminando em uma estabilização em torno de 0,103. Esse
comportamento sugere que a rede neural está sendo capaz de aprender a
relação entre as entradas e as saídas, porém ainda é possível considerar
melhorias no processo.

Com relação ao gráfico de sobreposição de saídas, foi possível constatar
que as saídas estimadas apresentaram uma boa aproximação em relação às
saídas desejadas, com discrepâncias mínimas entre elas. Esse resultado
evidencia que a rede neural está realizando uma aproximação satisfatória
da função que gera as saídas, contribuindo para aprimoramentos futuros
no processo de treinamento.

\hypertarget{segunda-questuxe3o}{%
\subsection{Segunda Questão}\label{segunda-questuxe3o}}

\begin{enumerate}
\def\labelenumi{\arabic{enumi})}
\setcounter{enumi}{1}
\tightlist
\item
  Agora, com este novo valor de época, altere o valor do eta para, por
  exemplo, 0.1. Análise o gráfico de erro e verifique se o ponto de
  diminuição abrupta (ponto em que faz o formato de ``cotovelo'' do
  gráfico) é anterior ou posterior ao do experimento anterior. Depois,
  refaça as tarefas pedidas no exercício 1 e compare os desempenhos. Por
  fim, conclua em que implica a alteração deste parâmetro e faça uma
  interpretação do que se espera acontecer se o valor deste parâmetro
  por aumentado (tendendo a 1).
\end{enumerate}

\begin{itemize}
\tightlist
\item
  Segue código condensado mudando a eta\textless- 0.01
\end{itemize}

\begin{Shaded}
\begin{Highlighting}[]
\NormalTok{N}\OtherTok{\textless{}{-}}\DecValTok{10}
\NormalTok{X}\OtherTok{\textless{}{-}}\FunctionTok{seq}\NormalTok{(}\DecValTok{1}\NormalTok{,N)}
\NormalTok{Y}\OtherTok{\textless{}{-}}\DecValTok{2}\SpecialCharTok{*}\NormalTok{X}\SpecialCharTok{+}\DecValTok{3}
\FunctionTok{plot}\NormalTok{(X,Y,}\AttributeTok{type=}\StringTok{"o"}\NormalTok{,}\AttributeTok{xlim=}\FunctionTok{c}\NormalTok{(}\DecValTok{0}\NormalTok{, }\DecValTok{10}\NormalTok{),}\AttributeTok{ylim=}\FunctionTok{c}\NormalTok{(}\DecValTok{0}\NormalTok{,}\DecValTok{25}\NormalTok{))}
\end{Highlighting}
\end{Shaded}

\includegraphics{trabalho_rna_aula_2_files/figure-latex/unnamed-chunk-7-1.pdf}

\begin{Shaded}
\begin{Highlighting}[]
\NormalTok{X}\OtherTok{\textless{}{-}}\NormalTok{X}\SpecialCharTok{/}\FunctionTok{max}\NormalTok{(X)}
\NormalTok{Y}\OtherTok{\textless{}{-}}\NormalTok{Y}\SpecialCharTok{/}\FunctionTok{max}\NormalTok{(Y)}
\NormalTok{X}\OtherTok{\textless{}{-}}\FunctionTok{t}\NormalTok{(}\FunctionTok{t}\NormalTok{(X))}
\NormalTok{bias}\OtherTok{\textless{}{-}}\DecValTok{1}
\NormalTok{X}\OtherTok{\textless{}{-}}\FunctionTok{cbind}\NormalTok{(X,bias)}
\FunctionTok{print}\NormalTok{(X)}
\end{Highlighting}
\end{Shaded}

\begin{verbatim}
##           bias
##  [1,] 0.1    1
##  [2,] 0.2    1
##  [3,] 0.3    1
##  [4,] 0.4    1
##  [5,] 0.5    1
##  [6,] 0.6    1
##  [7,] 0.7    1
##  [8,] 0.8    1
##  [9,] 0.9    1
## [10,] 1.0    1
\end{verbatim}

\begin{Shaded}
\begin{Highlighting}[]
\FunctionTok{print}\NormalTok{(}\FunctionTok{cbind}\NormalTok{(Y))}
\end{Highlighting}
\end{Shaded}

\begin{verbatim}
##               Y
##  [1,] 0.2173913
##  [2,] 0.3043478
##  [3,] 0.3913043
##  [4,] 0.4782609
##  [5,] 0.5652174
##  [6,] 0.6521739
##  [7,] 0.7391304
##  [8,] 0.8260870
##  [9,] 0.9130435
## [10,] 1.0000000
\end{verbatim}

\begin{Shaded}
\begin{Highlighting}[]
\NormalTok{epoca\_max}\OtherTok{\textless{}{-}}\DecValTok{300}
\NormalTok{eta}\OtherTok{\textless{}{-}}\FloatTok{0.1}
\NormalTok{W}\OtherTok{\textless{}{-}}\FunctionTok{c}\NormalTok{(}\DecValTok{0}\NormalTok{,}\DecValTok{1}\NormalTok{)}
\NormalTok{err\_iter}\OtherTok{\textless{}{-}}\FunctionTok{rep}\NormalTok{(}\DecValTok{0}\NormalTok{,N)}
\NormalTok{err\_epoc}\OtherTok{\textless{}{-}}\FunctionTok{rep}\NormalTok{(}\DecValTok{0}\NormalTok{,epoca\_max)}

\ControlFlowTok{for}\NormalTok{ (epoca }\ControlFlowTok{in} \DecValTok{1}\SpecialCharTok{:}\NormalTok{epoca\_max) \{}
  \ControlFlowTok{for}\NormalTok{ (i }\ControlFlowTok{in} \DecValTok{1}\SpecialCharTok{:}\NormalTok{N) \{}
\NormalTok{    v }\OtherTok{\textless{}{-}} \FunctionTok{sum}\NormalTok{(X[i,]}\SpecialCharTok{*}\NormalTok{W)}
\NormalTok{    erro }\OtherTok{\textless{}{-}}\NormalTok{ Y[i] }\SpecialCharTok{{-}}\NormalTok{ v}
\NormalTok{    delta }\OtherTok{\textless{}{-}}\NormalTok{ eta}\SpecialCharTok{*}\NormalTok{erro}\SpecialCharTok{*}\NormalTok{X[i,]}
\NormalTok{    W }\OtherTok{\textless{}{-}}\NormalTok{ W }\SpecialCharTok{+}\NormalTok{ delta}
\NormalTok{    err\_iter[i] }\OtherTok{\textless{}{-}} \FloatTok{0.5}\SpecialCharTok{*}\NormalTok{(erro}\SpecialCharTok{\^{}}\DecValTok{2}\NormalTok{)}
\NormalTok{  \}}
\NormalTok{  err\_epoc[epoca] }\OtherTok{\textless{}{-}} \FunctionTok{sum}\NormalTok{(err\_iter)}
\NormalTok{\}}
\FunctionTok{plot}\NormalTok{(err\_epoc, }\AttributeTok{type =} \StringTok{"line"}\NormalTok{, }\AttributeTok{xlab =} \StringTok{"época"}\NormalTok{, }\AttributeTok{ylab =} \StringTok{"erro quadrático médio"}\NormalTok{)}
\end{Highlighting}
\end{Shaded}

\begin{verbatim}
## Warning in plot.xy(xy, type, ...): plot type 'line' will be truncated to first
## character
\end{verbatim}

\includegraphics{trabalho_rna_aula_2_files/figure-latex/unnamed-chunk-9-1.pdf}

\begin{Shaded}
\begin{Highlighting}[]
\FunctionTok{print}\NormalTok{(W)}
\end{Highlighting}
\end{Shaded}

\begin{verbatim}
##                bias 
## 0.8695652 0.1304348
\end{verbatim}

\begin{Shaded}
\begin{Highlighting}[]
\NormalTok{Y\_estimado}\OtherTok{\textless{}{-}}\NormalTok{X[,}\DecValTok{1}\NormalTok{]}\SpecialCharTok{*}\NormalTok{W[}\DecValTok{1}\NormalTok{]}\SpecialCharTok{+}\NormalTok{W[}\DecValTok{2}\NormalTok{]}

\FunctionTok{plot}\NormalTok{(X[,}\DecValTok{1}\NormalTok{],Y,}\AttributeTok{type =}\StringTok{"n"}\NormalTok{)}
\FunctionTok{points}\NormalTok{(X[,}\DecValTok{1}\NormalTok{],Y,}\AttributeTok{pch=}\DecValTok{1}\NormalTok{,}\AttributeTok{type=}\StringTok{"o"}\NormalTok{,}\AttributeTok{col=}\StringTok{"1"}\NormalTok{)}
\FunctionTok{points}\NormalTok{(X[,}\DecValTok{1}\NormalTok{],Y\_estimado,}\AttributeTok{pch=}\DecValTok{16}\NormalTok{,}\AttributeTok{type=}\StringTok{"p"}\NormalTok{,}\AttributeTok{col=}\StringTok{"2"}\NormalTok{)}
\FunctionTok{legend}\NormalTok{(}\FloatTok{0.1}\NormalTok{,}\FloatTok{0.9}\NormalTok{,}\AttributeTok{legend=}\FunctionTok{c}\NormalTok{(}\StringTok{"Y\_desejado"}\NormalTok{,}\StringTok{"Y\_estimado"}\NormalTok{),}\AttributeTok{col=}\FunctionTok{c}\NormalTok{(}\DecValTok{1}\NormalTok{,}\DecValTok{2}\NormalTok{),}\AttributeTok{pch=}\FunctionTok{c}\NormalTok{(}\DecValTok{1}\NormalTok{,}\DecValTok{16}\NormalTok{))}
\end{Highlighting}
\end{Shaded}

\includegraphics{trabalho_rna_aula_2_files/figure-latex/unnamed-chunk-10-1.pdf}

\begin{Shaded}
\begin{Highlighting}[]
\FunctionTok{print}\NormalTok{(}\FunctionTok{paste}\NormalTok{(}\StringTok{"Erro final do treinamento"}\NormalTok{, err\_epoc[epoca\_max]))}
\end{Highlighting}
\end{Shaded}

\begin{verbatim}
## [1] "Erro final do treinamento 5.36757316768681e-18"
\end{verbatim}

\hypertarget{resposta-1}{%
\subsubsection{Resposta}\label{resposta-1}}

Com o valor de eta alterado para 0.1 e mantendo o número máximo de
épocas em 300, foi possível observar no gráfico de erro por época que o
ponto de diminuição abrupta ocorreu em torno da época 20, ou seja,
posterior ao experimento anterior com eta de 0.01.

Ao refazer as tarefas pedidas no exercício 1, verificou-se que a rede
neural apresentou um desempenho melhor em relação ao experimento
anterior, tendo alcançado um valor final de erro próximo de zero e uma
aproximação mais precisa entre as saídas desejadas e estimadas.

A alteração do valor do parâmetro eta afeta diretamente a taxa de
aprendizado da rede neural. Quando esse valor é aumentado, a rede pode
aprender mais rápido, mas também pode ocorrer o risco de que ela fique
presa em mínimos locais do erro e, portanto, não consiga aprender de
forma mais precisa. Isso pode afetar o desempenho geral da rede neural.

\hypertarget{terceira-questuxe3o}{%
\subsection{Terceira Questão}\label{terceira-questuxe3o}}

\begin{enumerate}
\def\labelenumi{\arabic{enumi})}
\setcounter{enumi}{2}
\tightlist
\item
  Por fim, refaça os mesmos ensaios e análises da questão 2, agora
  alterando o valor inicial dos pesos sinápticos para
  W\textless-c(0.1,1).
\end{enumerate}

\begin{itemize}
\tightlist
\item
  Segue código condensado mudando W para W\textless-c(0.1,1).
\end{itemize}

\begin{Shaded}
\begin{Highlighting}[]
\NormalTok{N}\OtherTok{\textless{}{-}}\DecValTok{10}
\NormalTok{X}\OtherTok{\textless{}{-}}\FunctionTok{seq}\NormalTok{(}\DecValTok{1}\NormalTok{,N)}
\NormalTok{Y}\OtherTok{\textless{}{-}}\DecValTok{2}\SpecialCharTok{*}\NormalTok{X}\SpecialCharTok{+}\DecValTok{3}
\FunctionTok{plot}\NormalTok{(X,Y,}\AttributeTok{type=}\StringTok{"o"}\NormalTok{,}\AttributeTok{xlim=}\FunctionTok{c}\NormalTok{(}\DecValTok{0}\NormalTok{, }\DecValTok{10}\NormalTok{),}\AttributeTok{ylim=}\FunctionTok{c}\NormalTok{(}\DecValTok{0}\NormalTok{,}\DecValTok{25}\NormalTok{))}
\end{Highlighting}
\end{Shaded}

\includegraphics{trabalho_rna_aula_2_files/figure-latex/unnamed-chunk-11-1.pdf}

\begin{Shaded}
\begin{Highlighting}[]
\NormalTok{X}\OtherTok{\textless{}{-}}\NormalTok{X}\SpecialCharTok{/}\FunctionTok{max}\NormalTok{(X)}
\NormalTok{Y}\OtherTok{\textless{}{-}}\NormalTok{Y}\SpecialCharTok{/}\FunctionTok{max}\NormalTok{(Y)}
\NormalTok{X}\OtherTok{\textless{}{-}}\FunctionTok{t}\NormalTok{(}\FunctionTok{t}\NormalTok{(X))}
\NormalTok{bias}\OtherTok{\textless{}{-}}\DecValTok{1}
\NormalTok{X}\OtherTok{\textless{}{-}}\FunctionTok{cbind}\NormalTok{(X,bias)}
\FunctionTok{print}\NormalTok{(X)}
\end{Highlighting}
\end{Shaded}

\begin{verbatim}
##           bias
##  [1,] 0.1    1
##  [2,] 0.2    1
##  [3,] 0.3    1
##  [4,] 0.4    1
##  [5,] 0.5    1
##  [6,] 0.6    1
##  [7,] 0.7    1
##  [8,] 0.8    1
##  [9,] 0.9    1
## [10,] 1.0    1
\end{verbatim}

\begin{Shaded}
\begin{Highlighting}[]
\FunctionTok{print}\NormalTok{(}\FunctionTok{cbind}\NormalTok{(Y))}
\end{Highlighting}
\end{Shaded}

\begin{verbatim}
##               Y
##  [1,] 0.2173913
##  [2,] 0.3043478
##  [3,] 0.3913043
##  [4,] 0.4782609
##  [5,] 0.5652174
##  [6,] 0.6521739
##  [7,] 0.7391304
##  [8,] 0.8260870
##  [9,] 0.9130435
## [10,] 1.0000000
\end{verbatim}

\begin{Shaded}
\begin{Highlighting}[]
\NormalTok{epoca\_max}\OtherTok{\textless{}{-}}\DecValTok{300}
\NormalTok{eta}\OtherTok{\textless{}{-}}\FloatTok{0.1}
\NormalTok{W}\OtherTok{\textless{}{-}}\FunctionTok{c}\NormalTok{(}\FloatTok{0.1}\NormalTok{,}\DecValTok{1}\NormalTok{)}
\NormalTok{err\_iter}\OtherTok{\textless{}{-}}\FunctionTok{rep}\NormalTok{(}\DecValTok{0}\NormalTok{,N)}
\NormalTok{err\_epoc}\OtherTok{\textless{}{-}}\FunctionTok{rep}\NormalTok{(}\DecValTok{0}\NormalTok{,epoca\_max)}

\ControlFlowTok{for}\NormalTok{ (epoca }\ControlFlowTok{in} \DecValTok{1}\SpecialCharTok{:}\NormalTok{epoca\_max) \{}
  \ControlFlowTok{for}\NormalTok{ (i }\ControlFlowTok{in} \DecValTok{1}\SpecialCharTok{:}\NormalTok{N) \{}
\NormalTok{    v }\OtherTok{\textless{}{-}} \FunctionTok{sum}\NormalTok{(X[i,]}\SpecialCharTok{*}\NormalTok{W)}
\NormalTok{    erro }\OtherTok{\textless{}{-}}\NormalTok{ Y[i] }\SpecialCharTok{{-}}\NormalTok{ v}
\NormalTok{    delta }\OtherTok{\textless{}{-}}\NormalTok{ eta}\SpecialCharTok{*}\NormalTok{erro}\SpecialCharTok{*}\NormalTok{X[i,]}
\NormalTok{    W }\OtherTok{\textless{}{-}}\NormalTok{ W }\SpecialCharTok{+}\NormalTok{ delta}
\NormalTok{    err\_iter[i] }\OtherTok{\textless{}{-}} \FloatTok{0.5}\SpecialCharTok{*}\NormalTok{(erro}\SpecialCharTok{\^{}}\DecValTok{2}\NormalTok{)}
\NormalTok{  \}}
\NormalTok{  err\_epoc[epoca] }\OtherTok{\textless{}{-}} \FunctionTok{sum}\NormalTok{(err\_iter)}
\NormalTok{\}}
\FunctionTok{plot}\NormalTok{(err\_epoc, }\AttributeTok{type =} \StringTok{"line"}\NormalTok{, }\AttributeTok{xlab =} \StringTok{"época"}\NormalTok{, }\AttributeTok{ylab =} \StringTok{"erro quadrático médio"}\NormalTok{)}
\end{Highlighting}
\end{Shaded}

\begin{verbatim}
## Warning in plot.xy(xy, type, ...): plot type 'line' will be truncated to first
## character
\end{verbatim}

\includegraphics{trabalho_rna_aula_2_files/figure-latex/unnamed-chunk-13-1.pdf}

\begin{Shaded}
\begin{Highlighting}[]
\FunctionTok{print}\NormalTok{(W)}
\end{Highlighting}
\end{Shaded}

\begin{verbatim}
##                bias 
## 0.8695652 0.1304348
\end{verbatim}

\begin{Shaded}
\begin{Highlighting}[]
\NormalTok{Y\_estimado}\OtherTok{\textless{}{-}}\NormalTok{X[,}\DecValTok{1}\NormalTok{]}\SpecialCharTok{*}\NormalTok{W[}\DecValTok{1}\NormalTok{]}\SpecialCharTok{+}\NormalTok{W[}\DecValTok{2}\NormalTok{]}

\FunctionTok{plot}\NormalTok{(X[,}\DecValTok{1}\NormalTok{],Y,}\AttributeTok{type =}\StringTok{"n"}\NormalTok{)}
\FunctionTok{points}\NormalTok{(X[,}\DecValTok{1}\NormalTok{],Y,}\AttributeTok{pch=}\DecValTok{1}\NormalTok{,}\AttributeTok{type=}\StringTok{"o"}\NormalTok{,}\AttributeTok{col=}\StringTok{"1"}\NormalTok{)}
\FunctionTok{points}\NormalTok{(X[,}\DecValTok{1}\NormalTok{],Y\_estimado,}\AttributeTok{pch=}\DecValTok{16}\NormalTok{,}\AttributeTok{type=}\StringTok{"p"}\NormalTok{,}\AttributeTok{col=}\StringTok{"2"}\NormalTok{)}
\FunctionTok{legend}\NormalTok{(}\FloatTok{0.1}\NormalTok{,}\FloatTok{0.9}\NormalTok{,}\AttributeTok{legend=}\FunctionTok{c}\NormalTok{(}\StringTok{"Y\_desejado"}\NormalTok{,}\StringTok{"Y\_estimado"}\NormalTok{),}\AttributeTok{col=}\FunctionTok{c}\NormalTok{(}\DecValTok{1}\NormalTok{,}\DecValTok{2}\NormalTok{),}\AttributeTok{pch=}\FunctionTok{c}\NormalTok{(}\DecValTok{1}\NormalTok{,}\DecValTok{16}\NormalTok{))}
\end{Highlighting}
\end{Shaded}

\includegraphics{trabalho_rna_aula_2_files/figure-latex/unnamed-chunk-14-1.pdf}

\begin{Shaded}
\begin{Highlighting}[]
\FunctionTok{print}\NormalTok{(}\FunctionTok{paste}\NormalTok{(}\StringTok{"Erro final do treinamento"}\NormalTok{, err\_epoc[epoca\_max]))}
\end{Highlighting}
\end{Shaded}

\begin{verbatim}
## [1] "Erro final do treinamento 4.56434939771144e-18"
\end{verbatim}

\hypertarget{resposta-2}{%
\subsubsection{Resposta}\label{resposta-2}}

Ao alterar o valor inicial dos pesos sinápticos para W \textless- c(0.1,
1) e manter o valor de eta em 0.1 e o número máximo de épocas em 300,
foi possível observar que o ponto de diminuição abrupta no gráfico de
erro por época ocorreu em torno da época 20, similar ao exercício
anterior.

Ao refazer as tarefas pedidas no exercício 1 e comparar com os
resultados obtidos nos experimentos anteriores, verificou-se que a rede
neural teve um desempenho levemente inferior ao alcançado no experimento
com W \textless- c(0, 1). O valor final de erro foi maior e a
aproximação entre as saídas desejadas e estimadas apresentou
discrepâncias mais evidentes.

A alteração do valor inicial dos pesos sinápticos pode influenciar
significativamente o processo de treinamento da rede neural, pois esse
parâmetro define a posição inicial dos pesos. Um valor inadequado pode
levar a um comportamento oscilatório na atualização dos pesos e
dificultar o processo de convergência.

\hypertarget{segunda-parte}{%
\section{Segunda Parte}\label{segunda-parte}}

\hypertarget{caso-1---dataset-and}{%
\subsection{Caso 1 - dataset AND}\label{caso-1---dataset-and}}

\hypertarget{a-w-c0.10.11}{%
\subsubsection{a) W\textless-c(0.1,0.1,1)}\label{a-w-c0.10.11}}

\begin{Shaded}
\begin{Highlighting}[]
\CommentTok{\# Set de variáveis}
\NormalTok{W}\OtherTok{\textless{}{-}}\FunctionTok{c}\NormalTok{(}\FloatTok{0.1}\NormalTok{,}\FloatTok{0.1}\NormalTok{,}\DecValTok{1}\NormalTok{)}
\NormalTok{eta}\OtherTok{\textless{}{-}}\FloatTok{0.1}
\NormalTok{epoca\_max}\OtherTok{\textless{}{-}}\DecValTok{20}

\CommentTok{\# Matriz de inputs e Bias}
\NormalTok{x1}\OtherTok{\textless{}{-}}\FunctionTok{c}\NormalTok{(}\DecValTok{0}\NormalTok{,}\DecValTok{0}\NormalTok{,}\DecValTok{1}\NormalTok{)}
\NormalTok{x2}\OtherTok{\textless{}{-}}\FunctionTok{c}\NormalTok{(}\DecValTok{0}\NormalTok{,}\DecValTok{1}\NormalTok{,}\DecValTok{1}\NormalTok{)}
\NormalTok{x3}\OtherTok{\textless{}{-}}\FunctionTok{c}\NormalTok{(}\DecValTok{1}\NormalTok{,}\DecValTok{0}\NormalTok{,}\DecValTok{1}\NormalTok{)}
\NormalTok{x4}\OtherTok{\textless{}{-}}\FunctionTok{c}\NormalTok{(}\DecValTok{1}\NormalTok{,}\DecValTok{1}\NormalTok{,}\DecValTok{1}\NormalTok{)}

\CommentTok{\# Bind de inputs e Bias}
\NormalTok{X}\OtherTok{\textless{}{-}}\FunctionTok{rbind}\NormalTok{(x1,x2,x3,x4)}
\FunctionTok{print}\NormalTok{(X)}
\end{Highlighting}
\end{Shaded}

\begin{verbatim}
##    [,1] [,2] [,3]
## x1    0    0    1
## x2    0    1    1
## x3    1    0    1
## x4    1    1    1
\end{verbatim}

\begin{Shaded}
\begin{Highlighting}[]
\CommentTok{\# Saída desejada}
\NormalTok{Y}\OtherTok{\textless{}{-}}\FunctionTok{c}\NormalTok{(}\DecValTok{0}\NormalTok{,}\DecValTok{0}\NormalTok{,}\DecValTok{0}\NormalTok{,}\DecValTok{1}\NormalTok{)}
\FunctionTok{print}\NormalTok{(Y)}
\end{Highlighting}
\end{Shaded}

\begin{verbatim}
## [1] 0 0 0 1
\end{verbatim}

\begin{Shaded}
\begin{Highlighting}[]
\CommentTok{\# Visualização dos dados}
\FunctionTok{plot}\NormalTok{(X,}\AttributeTok{type=}\StringTok{"n"}\NormalTok{)}
\FunctionTok{points}\NormalTok{(x1[}\DecValTok{1}\NormalTok{],x1[}\DecValTok{2}\NormalTok{],}\AttributeTok{pch=}\FunctionTok{c}\NormalTok{(}\DecValTok{19}\NormalTok{),}\AttributeTok{cex=}\DecValTok{2}\NormalTok{,}\AttributeTok{col=}\StringTok{"2"}\NormalTok{)}
\FunctionTok{points}\NormalTok{(x2[}\DecValTok{1}\NormalTok{],x2[}\DecValTok{2}\NormalTok{],}\AttributeTok{pch=}\FunctionTok{c}\NormalTok{(}\DecValTok{19}\NormalTok{),}\AttributeTok{cex=}\DecValTok{2}\NormalTok{,}\AttributeTok{col=}\StringTok{"2"}\NormalTok{)}
\FunctionTok{legend}\NormalTok{(}\FloatTok{0.4}\NormalTok{,}\FloatTok{0.6}\NormalTok{,}\AttributeTok{legend=}\FunctionTok{c}\NormalTok{(}\StringTok{"0"}\NormalTok{,}\StringTok{"1"}\NormalTok{),}\AttributeTok{col=}\FunctionTok{c}\NormalTok{(}\DecValTok{2}\NormalTok{,}\DecValTok{1}\NormalTok{), }\AttributeTok{pch=}\FunctionTok{c}\NormalTok{(}\DecValTok{19}\NormalTok{,}\DecValTok{15}\NormalTok{))}
\end{Highlighting}
\end{Shaded}

\includegraphics{trabalho_rna_aula_2_files/figure-latex/unnamed-chunk-15-1.pdf}

\begin{Shaded}
\begin{Highlighting}[]
\CommentTok{\#Erros}
\NormalTok{erro\_ite}\OtherTok{\textless{}{-}}\FunctionTok{rep}\NormalTok{(}\DecValTok{0}\NormalTok{,}\FunctionTok{dim}\NormalTok{(X)[}\DecValTok{1}\NormalTok{])}
\NormalTok{erro\_total}\OtherTok{\textless{}{-}}\FunctionTok{rep}\NormalTok{(}\DecValTok{0}\NormalTok{,epoca\_max)}

\CommentTok{\# Treinamento}
\ControlFlowTok{for}\NormalTok{(epoca }\ControlFlowTok{in} \DecValTok{1}\SpecialCharTok{:}\NormalTok{epoca\_max)\{}
  \ControlFlowTok{for}\NormalTok{(i }\ControlFlowTok{in} \DecValTok{1}\SpecialCharTok{:}\FunctionTok{dim}\NormalTok{(X)[}\DecValTok{1}\NormalTok{]) \{}
\NormalTok{    v}\OtherTok{\textless{}{-}}\FunctionTok{sum}\NormalTok{(X[i,]}\SpecialCharTok{*}\NormalTok{W)}
    \ControlFlowTok{if}\NormalTok{(v}\SpecialCharTok{\textgreater{}}\DecValTok{0}\NormalTok{)\{}
\NormalTok{      y\_calc}\OtherTok{\textless{}{-}}\DecValTok{1}
\NormalTok{    \}}\ControlFlowTok{else}\NormalTok{\{}
\NormalTok{      y\_calc}\OtherTok{\textless{}{-}}\DecValTok{0}
\NormalTok{    \}}
\NormalTok{    erro}\OtherTok{\textless{}{-}}\NormalTok{Y[i]}\SpecialCharTok{{-}}\NormalTok{y\_calc}
\NormalTok{    delta}\OtherTok{\textless{}{-}}\NormalTok{(eta}\SpecialCharTok{*}\NormalTok{erro}\SpecialCharTok{*}\NormalTok{X[i,])}
\NormalTok{    W}\OtherTok{\textless{}{-}}\NormalTok{W}\SpecialCharTok{+}\NormalTok{delta}
\NormalTok{    erro\_ite[i]}\OtherTok{\textless{}{-}}\NormalTok{erro}\SpecialCharTok{\^{}}\DecValTok{2}
\NormalTok{  \}}
\NormalTok{  erro\_total[epoca]}\OtherTok{\textless{}{-}}\FunctionTok{sum}\NormalTok{(erro\_ite)}
  \ControlFlowTok{if}\NormalTok{(}\FunctionTok{sum}\NormalTok{(erro\_ite)}\SpecialCharTok{==}\DecValTok{0}\NormalTok{)\{}
    \ControlFlowTok{break}
\NormalTok{  \}}
\NormalTok{\}}

\FunctionTok{print}\NormalTok{(}\FunctionTok{paste}\NormalTok{(}\StringTok{"Número de épocas usadas no treinamento"}\NormalTok{,epoca,}\StringTok{"de um máximo de"}\NormalTok{,epoca\_max))}
\end{Highlighting}
\end{Shaded}

\begin{verbatim}
## [1] "Número de épocas usadas no treinamento 11 de um máximo de 20"
\end{verbatim}

\begin{Shaded}
\begin{Highlighting}[]
\CommentTok{\# Peso Sináptico}
\FunctionTok{print}\NormalTok{(W)}
\end{Highlighting}
\end{Shaded}

\begin{verbatim}
## [1]  0.2  0.1 -0.3
\end{verbatim}

\begin{Shaded}
\begin{Highlighting}[]
\CommentTok{\# Erro total}
\FunctionTok{print}\NormalTok{(erro\_total)}
\end{Highlighting}
\end{Shaded}

\begin{verbatim}
##  [1] 3 3 4 3 3 2 3 3 2 1 0 0 0 0 0 0 0 0 0 0
\end{verbatim}

\begin{Shaded}
\begin{Highlighting}[]
\FunctionTok{plot}\NormalTok{(erro\_total,}\AttributeTok{type=}\StringTok{"l"}\NormalTok{)}
\end{Highlighting}
\end{Shaded}

\includegraphics{trabalho_rna_aula_2_files/figure-latex/unnamed-chunk-15-2.pdf}

\begin{Shaded}
\begin{Highlighting}[]
\CommentTok{\# Gráfico do Conjunto de Treinamento}
\FunctionTok{plot}\NormalTok{(X,}\AttributeTok{type=}\StringTok{"n"}\NormalTok{,}\AttributeTok{xlim=}\FunctionTok{c}\NormalTok{(}\SpecialCharTok{{-}}\FloatTok{1.5}\NormalTok{,}\FloatTok{1.5}\NormalTok{ ), }\AttributeTok{ylim=}\FunctionTok{c}\NormalTok{(}\SpecialCharTok{{-}}\FloatTok{1.5}\NormalTok{,}\FloatTok{1.5}\NormalTok{))}
\FunctionTok{points}\NormalTok{(x1[}\DecValTok{1}\NormalTok{],x1[}\DecValTok{2}\NormalTok{],}\AttributeTok{pch=}\FunctionTok{c}\NormalTok{(}\DecValTok{19}\NormalTok{),}\AttributeTok{cex=}\DecValTok{2}\NormalTok{,}\AttributeTok{col=}\StringTok{"2"}\NormalTok{)}
\FunctionTok{points}\NormalTok{(x2[}\DecValTok{1}\NormalTok{],x2[}\DecValTok{2}\NormalTok{],}\AttributeTok{pch=}\FunctionTok{c}\NormalTok{(}\DecValTok{19}\NormalTok{),}\AttributeTok{cex=}\DecValTok{2}\NormalTok{,}\AttributeTok{col=}\StringTok{"2"}\NormalTok{)}
\FunctionTok{points}\NormalTok{(x3[}\DecValTok{1}\NormalTok{],x3[}\DecValTok{2}\NormalTok{],}\AttributeTok{pch=}\FunctionTok{c}\NormalTok{(}\DecValTok{19}\NormalTok{),}\AttributeTok{cex=}\DecValTok{2}\NormalTok{,}\AttributeTok{col=}\StringTok{"2"}\NormalTok{)}
\FunctionTok{points}\NormalTok{(x4[}\DecValTok{1}\NormalTok{],x4[}\DecValTok{2}\NormalTok{],}\AttributeTok{pch=}\FunctionTok{c}\NormalTok{(}\DecValTok{15}\NormalTok{),}\AttributeTok{cex=}\DecValTok{2}\NormalTok{,}\AttributeTok{col=}\StringTok{"1"}\NormalTok{)}
\FunctionTok{legend}\NormalTok{(}\FloatTok{0.4}\NormalTok{,}\FloatTok{0.6}\NormalTok{,}\AttributeTok{legend=}\FunctionTok{c}\NormalTok{(}\StringTok{"0"}\NormalTok{,}\StringTok{"1"}\NormalTok{),}\AttributeTok{col=}\FunctionTok{c}\NormalTok{(}\DecValTok{2}\NormalTok{,}\DecValTok{1}\NormalTok{), }\AttributeTok{pch=}\FunctionTok{c}\NormalTok{(}\DecValTok{19}\NormalTok{,}\DecValTok{15}\NormalTok{))}

\NormalTok{x\_1}\OtherTok{\textless{}{-}} \SpecialCharTok{{-}}\NormalTok{(W[}\DecValTok{3}\NormalTok{]}\SpecialCharTok{/}\NormalTok{W[}\DecValTok{1}\NormalTok{])}

\NormalTok{x\_2}\OtherTok{\textless{}{-}} \SpecialCharTok{{-}}\NormalTok{(W[}\DecValTok{3}\NormalTok{]}\SpecialCharTok{/}\NormalTok{W[}\DecValTok{2}\NormalTok{])}

\FunctionTok{segments}\NormalTok{(x\_1,}\DecValTok{0}\NormalTok{,}\DecValTok{0}\NormalTok{,x\_2)}
\end{Highlighting}
\end{Shaded}

\includegraphics{trabalho_rna_aula_2_files/figure-latex/unnamed-chunk-15-3.pdf}

Embora o comportamento do aprendizado seja semelhante quando o erro
tende a zero, a oscilação do gráfico de erro indica que o processo de
aprendizagem como um todo foi diferente.

\hypertarget{b-eta-0.01}{%
\subsubsection{b) eta\textless-0.01}\label{b-eta-0.01}}

\begin{Shaded}
\begin{Highlighting}[]
\CommentTok{\# Set de variáveis}
\NormalTok{W}\OtherTok{\textless{}{-}}\FunctionTok{c}\NormalTok{(}\FloatTok{0.1}\NormalTok{,}\FloatTok{0.1}\NormalTok{,}\DecValTok{1}\NormalTok{)}
\NormalTok{eta}\OtherTok{\textless{}{-}}\FloatTok{0.01}
\NormalTok{epoca\_max}\OtherTok{\textless{}{-}}\DecValTok{20}

\CommentTok{\# Matriz de inputs e Bias}
\NormalTok{x1}\OtherTok{\textless{}{-}}\FunctionTok{c}\NormalTok{(}\DecValTok{0}\NormalTok{,}\DecValTok{0}\NormalTok{,}\DecValTok{1}\NormalTok{)}
\NormalTok{x2}\OtherTok{\textless{}{-}}\FunctionTok{c}\NormalTok{(}\DecValTok{0}\NormalTok{,}\DecValTok{1}\NormalTok{,}\DecValTok{1}\NormalTok{)}
\NormalTok{x3}\OtherTok{\textless{}{-}}\FunctionTok{c}\NormalTok{(}\DecValTok{1}\NormalTok{,}\DecValTok{0}\NormalTok{,}\DecValTok{1}\NormalTok{)}
\NormalTok{x4}\OtherTok{\textless{}{-}}\FunctionTok{c}\NormalTok{(}\DecValTok{1}\NormalTok{,}\DecValTok{1}\NormalTok{,}\DecValTok{1}\NormalTok{)}

\CommentTok{\# Bind de inputs e Bias}
\NormalTok{X}\OtherTok{\textless{}{-}}\FunctionTok{rbind}\NormalTok{(x1,x2,x3,x4)}
\FunctionTok{print}\NormalTok{(X)}
\end{Highlighting}
\end{Shaded}

\begin{verbatim}
##    [,1] [,2] [,3]
## x1    0    0    1
## x2    0    1    1
## x3    1    0    1
## x4    1    1    1
\end{verbatim}

\begin{Shaded}
\begin{Highlighting}[]
\CommentTok{\# Saída desejada}
\NormalTok{Y}\OtherTok{\textless{}{-}}\FunctionTok{c}\NormalTok{(}\DecValTok{0}\NormalTok{,}\DecValTok{0}\NormalTok{,}\DecValTok{0}\NormalTok{,}\DecValTok{1}\NormalTok{)}
\FunctionTok{print}\NormalTok{(Y)}
\end{Highlighting}
\end{Shaded}

\begin{verbatim}
## [1] 0 0 0 1
\end{verbatim}

\begin{Shaded}
\begin{Highlighting}[]
\CommentTok{\# Visualização dos dados}
\FunctionTok{plot}\NormalTok{(X,}\AttributeTok{type=}\StringTok{"n"}\NormalTok{)}
\FunctionTok{points}\NormalTok{(x1[}\DecValTok{1}\NormalTok{],x1[}\DecValTok{2}\NormalTok{],}\AttributeTok{pch=}\FunctionTok{c}\NormalTok{(}\DecValTok{19}\NormalTok{),}\AttributeTok{cex=}\DecValTok{2}\NormalTok{,}\AttributeTok{col=}\StringTok{"2"}\NormalTok{)}
\FunctionTok{points}\NormalTok{(x2[}\DecValTok{1}\NormalTok{],x2[}\DecValTok{2}\NormalTok{],}\AttributeTok{pch=}\FunctionTok{c}\NormalTok{(}\DecValTok{19}\NormalTok{),}\AttributeTok{cex=}\DecValTok{2}\NormalTok{,}\AttributeTok{col=}\StringTok{"2"}\NormalTok{)}
\FunctionTok{legend}\NormalTok{(}\FloatTok{0.4}\NormalTok{,}\FloatTok{0.6}\NormalTok{,}\AttributeTok{legend=}\FunctionTok{c}\NormalTok{(}\StringTok{"0"}\NormalTok{,}\StringTok{"1"}\NormalTok{),}\AttributeTok{col=}\FunctionTok{c}\NormalTok{(}\DecValTok{2}\NormalTok{,}\DecValTok{1}\NormalTok{), }\AttributeTok{pch=}\FunctionTok{c}\NormalTok{(}\DecValTok{19}\NormalTok{,}\DecValTok{15}\NormalTok{))}
\end{Highlighting}
\end{Shaded}

\includegraphics{trabalho_rna_aula_2_files/figure-latex/unnamed-chunk-16-1.pdf}

\begin{Shaded}
\begin{Highlighting}[]
\CommentTok{\#Erros}
\NormalTok{erro\_ite}\OtherTok{\textless{}{-}}\FunctionTok{rep}\NormalTok{(}\DecValTok{0}\NormalTok{,}\FunctionTok{dim}\NormalTok{(X)[}\DecValTok{1}\NormalTok{])}
\NormalTok{erro\_total}\OtherTok{\textless{}{-}}\FunctionTok{rep}\NormalTok{(}\DecValTok{0}\NormalTok{,epoca\_max)}

\CommentTok{\# Treinamento}
\ControlFlowTok{for}\NormalTok{(epoca }\ControlFlowTok{in} \DecValTok{1}\SpecialCharTok{:}\NormalTok{epoca\_max)\{}
  \ControlFlowTok{for}\NormalTok{(i }\ControlFlowTok{in} \DecValTok{1}\SpecialCharTok{:}\FunctionTok{dim}\NormalTok{(X)[}\DecValTok{1}\NormalTok{]) \{}
\NormalTok{    v}\OtherTok{\textless{}{-}}\FunctionTok{sum}\NormalTok{(X[i,]}\SpecialCharTok{*}\NormalTok{W)}
    \ControlFlowTok{if}\NormalTok{(v}\SpecialCharTok{\textgreater{}}\DecValTok{0}\NormalTok{)\{}
\NormalTok{      y\_calc}\OtherTok{\textless{}{-}}\DecValTok{1}
\NormalTok{    \}}\ControlFlowTok{else}\NormalTok{\{}
\NormalTok{      y\_calc}\OtherTok{\textless{}{-}}\DecValTok{0}
\NormalTok{    \}}
\NormalTok{    erro}\OtherTok{\textless{}{-}}\NormalTok{Y[i]}\SpecialCharTok{{-}}\NormalTok{y\_calc}
\NormalTok{    delta}\OtherTok{\textless{}{-}}\NormalTok{(eta}\SpecialCharTok{*}\NormalTok{erro}\SpecialCharTok{*}\NormalTok{X[i,])}
\NormalTok{    W}\OtherTok{\textless{}{-}}\NormalTok{W}\SpecialCharTok{+}\NormalTok{delta}
\NormalTok{    erro\_ite[i]}\OtherTok{\textless{}{-}}\NormalTok{erro}\SpecialCharTok{\^{}}\DecValTok{2}
\NormalTok{  \}}
\NormalTok{  erro\_total[epoca]}\OtherTok{\textless{}{-}}\FunctionTok{sum}\NormalTok{(erro\_ite)}
  \ControlFlowTok{if}\NormalTok{(}\FunctionTok{sum}\NormalTok{(erro\_ite)}\SpecialCharTok{==}\DecValTok{0}\NormalTok{)\{}
    \ControlFlowTok{break}
\NormalTok{  \}}
\NormalTok{\}}

\FunctionTok{print}\NormalTok{(}\FunctionTok{paste}\NormalTok{(}\StringTok{"Número de épocas usadas no treinamento"}\NormalTok{,epoca,}\StringTok{"de um máximo de"}\NormalTok{,epoca\_max))}
\end{Highlighting}
\end{Shaded}

\begin{verbatim}
## [1] "Número de épocas usadas no treinamento 20 de um máximo de 20"
\end{verbatim}

\begin{Shaded}
\begin{Highlighting}[]
\CommentTok{\# Peso Sináptico}
\FunctionTok{print}\NormalTok{(W)}
\end{Highlighting}
\end{Shaded}

\begin{verbatim}
## [1] -0.1 -0.1  0.4
\end{verbatim}

\begin{Shaded}
\begin{Highlighting}[]
\CommentTok{\# Erro total}
\FunctionTok{print}\NormalTok{(erro\_total)}
\end{Highlighting}
\end{Shaded}

\begin{verbatim}
##  [1] 3 3 3 3 3 3 3 3 3 3 3 3 3 3 3 3 3 3 3 3
\end{verbatim}

\begin{Shaded}
\begin{Highlighting}[]
\FunctionTok{plot}\NormalTok{(erro\_total,}\AttributeTok{type=}\StringTok{"l"}\NormalTok{)}
\end{Highlighting}
\end{Shaded}

\includegraphics{trabalho_rna_aula_2_files/figure-latex/unnamed-chunk-16-2.pdf}

\begin{Shaded}
\begin{Highlighting}[]
\CommentTok{\# Gráfico do Conjunto de Treinamento}
\FunctionTok{plot}\NormalTok{(X,}\AttributeTok{type=}\StringTok{"n"}\NormalTok{,}\AttributeTok{xlim=}\FunctionTok{c}\NormalTok{(}\SpecialCharTok{{-}}\FloatTok{1.5}\NormalTok{,}\FloatTok{1.5}\NormalTok{ ), }\AttributeTok{ylim=}\FunctionTok{c}\NormalTok{(}\SpecialCharTok{{-}}\FloatTok{1.5}\NormalTok{,}\FloatTok{1.5}\NormalTok{))}
\FunctionTok{points}\NormalTok{(x1[}\DecValTok{1}\NormalTok{],x1[}\DecValTok{2}\NormalTok{],}\AttributeTok{pch=}\FunctionTok{c}\NormalTok{(}\DecValTok{19}\NormalTok{),}\AttributeTok{cex=}\DecValTok{2}\NormalTok{,}\AttributeTok{col=}\StringTok{"2"}\NormalTok{)}
\FunctionTok{points}\NormalTok{(x2[}\DecValTok{1}\NormalTok{],x2[}\DecValTok{2}\NormalTok{],}\AttributeTok{pch=}\FunctionTok{c}\NormalTok{(}\DecValTok{19}\NormalTok{),}\AttributeTok{cex=}\DecValTok{2}\NormalTok{,}\AttributeTok{col=}\StringTok{"2"}\NormalTok{)}
\FunctionTok{points}\NormalTok{(x3[}\DecValTok{1}\NormalTok{],x3[}\DecValTok{2}\NormalTok{],}\AttributeTok{pch=}\FunctionTok{c}\NormalTok{(}\DecValTok{19}\NormalTok{),}\AttributeTok{cex=}\DecValTok{2}\NormalTok{,}\AttributeTok{col=}\StringTok{"2"}\NormalTok{)}
\FunctionTok{points}\NormalTok{(x4[}\DecValTok{1}\NormalTok{],x4[}\DecValTok{2}\NormalTok{],}\AttributeTok{pch=}\FunctionTok{c}\NormalTok{(}\DecValTok{15}\NormalTok{),}\AttributeTok{cex=}\DecValTok{2}\NormalTok{,}\AttributeTok{col=}\StringTok{"1"}\NormalTok{)}
\FunctionTok{legend}\NormalTok{(}\FloatTok{0.4}\NormalTok{,}\FloatTok{0.6}\NormalTok{,}\AttributeTok{legend=}\FunctionTok{c}\NormalTok{(}\StringTok{"0"}\NormalTok{,}\StringTok{"1"}\NormalTok{),}\AttributeTok{col=}\FunctionTok{c}\NormalTok{(}\DecValTok{2}\NormalTok{,}\DecValTok{1}\NormalTok{), }\AttributeTok{pch=}\FunctionTok{c}\NormalTok{(}\DecValTok{19}\NormalTok{,}\DecValTok{15}\NormalTok{))}

\NormalTok{x\_1}\OtherTok{\textless{}{-}} \SpecialCharTok{{-}}\NormalTok{(W[}\DecValTok{3}\NormalTok{]}\SpecialCharTok{/}\NormalTok{W[}\DecValTok{1}\NormalTok{])}

\NormalTok{x\_2}\OtherTok{\textless{}{-}} \SpecialCharTok{{-}}\NormalTok{(W[}\DecValTok{3}\NormalTok{]}\SpecialCharTok{/}\NormalTok{W[}\DecValTok{2}\NormalTok{])}

\FunctionTok{segments}\NormalTok{(x\_1,}\DecValTok{0}\NormalTok{,}\DecValTok{0}\NormalTok{,x\_2)}
\end{Highlighting}
\end{Shaded}

\includegraphics{trabalho_rna_aula_2_files/figure-latex/unnamed-chunk-16-3.pdf}

Neste caso o neurônio não aprende no número de épocas programado (20),
pois o gráfico de erro está estagnado em 3.0.

\hypertarget{c-w-c0.101}{%
\subsubsection{c) W\textless-c(0.1,0,1)}\label{c-w-c0.101}}

\begin{Shaded}
\begin{Highlighting}[]
\CommentTok{\# Set de variáveis}
\NormalTok{W}\OtherTok{\textless{}{-}}\FunctionTok{c}\NormalTok{(}\FloatTok{0.1}\NormalTok{,}\DecValTok{0}\NormalTok{,}\DecValTok{1}\NormalTok{)}
\NormalTok{eta}\OtherTok{\textless{}{-}}\FloatTok{0.01}
\NormalTok{epoca\_max}\OtherTok{\textless{}{-}}\DecValTok{80}

\CommentTok{\# Matriz de inputs e Bias}
\NormalTok{x1}\OtherTok{\textless{}{-}}\FunctionTok{c}\NormalTok{(}\DecValTok{0}\NormalTok{,}\DecValTok{0}\NormalTok{,}\DecValTok{1}\NormalTok{)}
\NormalTok{x2}\OtherTok{\textless{}{-}}\FunctionTok{c}\NormalTok{(}\DecValTok{0}\NormalTok{,}\DecValTok{1}\NormalTok{,}\DecValTok{1}\NormalTok{)}
\NormalTok{x3}\OtherTok{\textless{}{-}}\FunctionTok{c}\NormalTok{(}\DecValTok{1}\NormalTok{,}\DecValTok{0}\NormalTok{,}\DecValTok{1}\NormalTok{)}
\NormalTok{x4}\OtherTok{\textless{}{-}}\FunctionTok{c}\NormalTok{(}\DecValTok{1}\NormalTok{,}\DecValTok{1}\NormalTok{,}\DecValTok{1}\NormalTok{)}

\CommentTok{\# Bind de inputs e Bias}
\NormalTok{X}\OtherTok{\textless{}{-}}\FunctionTok{rbind}\NormalTok{(x1,x2,x3,x4)}
\FunctionTok{print}\NormalTok{(X)}
\end{Highlighting}
\end{Shaded}

\begin{verbatim}
##    [,1] [,2] [,3]
## x1    0    0    1
## x2    0    1    1
## x3    1    0    1
## x4    1    1    1
\end{verbatim}

\begin{Shaded}
\begin{Highlighting}[]
\CommentTok{\# Saída desejada}
\NormalTok{Y}\OtherTok{\textless{}{-}}\FunctionTok{c}\NormalTok{(}\DecValTok{0}\NormalTok{,}\DecValTok{0}\NormalTok{,}\DecValTok{0}\NormalTok{,}\DecValTok{1}\NormalTok{)}
\FunctionTok{print}\NormalTok{(Y)}
\end{Highlighting}
\end{Shaded}

\begin{verbatim}
## [1] 0 0 0 1
\end{verbatim}

\begin{Shaded}
\begin{Highlighting}[]
\CommentTok{\# Visualização dos dados}
\FunctionTok{plot}\NormalTok{(X,}\AttributeTok{type=}\StringTok{"n"}\NormalTok{)}
\FunctionTok{points}\NormalTok{(x1[}\DecValTok{1}\NormalTok{],x1[}\DecValTok{2}\NormalTok{],}\AttributeTok{pch=}\FunctionTok{c}\NormalTok{(}\DecValTok{19}\NormalTok{),}\AttributeTok{cex=}\DecValTok{2}\NormalTok{,}\AttributeTok{col=}\StringTok{"2"}\NormalTok{)}
\FunctionTok{points}\NormalTok{(x2[}\DecValTok{1}\NormalTok{],x2[}\DecValTok{2}\NormalTok{],}\AttributeTok{pch=}\FunctionTok{c}\NormalTok{(}\DecValTok{19}\NormalTok{),}\AttributeTok{cex=}\DecValTok{2}\NormalTok{,}\AttributeTok{col=}\StringTok{"2"}\NormalTok{)}
\FunctionTok{legend}\NormalTok{(}\FloatTok{0.4}\NormalTok{,}\FloatTok{0.6}\NormalTok{,}\AttributeTok{legend=}\FunctionTok{c}\NormalTok{(}\StringTok{"0"}\NormalTok{,}\StringTok{"1"}\NormalTok{),}\AttributeTok{col=}\FunctionTok{c}\NormalTok{(}\DecValTok{2}\NormalTok{,}\DecValTok{1}\NormalTok{), }\AttributeTok{pch=}\FunctionTok{c}\NormalTok{(}\DecValTok{19}\NormalTok{,}\DecValTok{15}\NormalTok{))}
\end{Highlighting}
\end{Shaded}

\includegraphics{trabalho_rna_aula_2_files/figure-latex/unnamed-chunk-17-1.pdf}

\begin{Shaded}
\begin{Highlighting}[]
\CommentTok{\#Erros}
\NormalTok{erro\_ite}\OtherTok{\textless{}{-}}\FunctionTok{rep}\NormalTok{(}\DecValTok{0}\NormalTok{,}\FunctionTok{dim}\NormalTok{(X)[}\DecValTok{1}\NormalTok{])}
\NormalTok{erro\_total}\OtherTok{\textless{}{-}}\FunctionTok{rep}\NormalTok{(}\DecValTok{0}\NormalTok{,epoca\_max)}

\CommentTok{\# Treinamento}
\ControlFlowTok{for}\NormalTok{(epoca }\ControlFlowTok{in} \DecValTok{1}\SpecialCharTok{:}\NormalTok{epoca\_max)\{}
  \ControlFlowTok{for}\NormalTok{(i }\ControlFlowTok{in} \DecValTok{1}\SpecialCharTok{:}\FunctionTok{dim}\NormalTok{(X)[}\DecValTok{1}\NormalTok{]) \{}
\NormalTok{    v}\OtherTok{\textless{}{-}}\FunctionTok{sum}\NormalTok{(X[i,]}\SpecialCharTok{*}\NormalTok{W)}
    \ControlFlowTok{if}\NormalTok{(v}\SpecialCharTok{\textgreater{}}\DecValTok{0}\NormalTok{)\{}
\NormalTok{      y\_calc}\OtherTok{\textless{}{-}}\DecValTok{1}
\NormalTok{    \}}\ControlFlowTok{else}\NormalTok{\{}
\NormalTok{      y\_calc}\OtherTok{\textless{}{-}}\DecValTok{0}
\NormalTok{    \}}
\NormalTok{    erro}\OtherTok{\textless{}{-}}\NormalTok{Y[i]}\SpecialCharTok{{-}}\NormalTok{y\_calc}
\NormalTok{    delta}\OtherTok{\textless{}{-}}\NormalTok{(eta}\SpecialCharTok{*}\NormalTok{erro}\SpecialCharTok{*}\NormalTok{X[i,])}
\NormalTok{    W}\OtherTok{\textless{}{-}}\NormalTok{W}\SpecialCharTok{+}\NormalTok{delta}
\NormalTok{    erro\_ite[i]}\OtherTok{\textless{}{-}}\NormalTok{erro}\SpecialCharTok{\^{}}\DecValTok{2}
\NormalTok{  \}}
\NormalTok{  erro\_total[epoca]}\OtherTok{\textless{}{-}}\FunctionTok{sum}\NormalTok{(erro\_ite)}
  \ControlFlowTok{if}\NormalTok{(}\FunctionTok{sum}\NormalTok{(erro\_ite)}\SpecialCharTok{==}\DecValTok{0}\NormalTok{)\{}
    \ControlFlowTok{break}
\NormalTok{  \}}
\NormalTok{\}}

\FunctionTok{print}\NormalTok{(}\FunctionTok{paste}\NormalTok{(}\StringTok{"Número de épocas usadas no treinamento"}\NormalTok{,epoca,}\StringTok{"de um máximo de"}\NormalTok{,epoca\_max))}
\end{Highlighting}
\end{Shaded}

\begin{verbatim}
## [1] "Número de épocas usadas no treinamento 61 de um máximo de 80"
\end{verbatim}

\begin{Shaded}
\begin{Highlighting}[]
\CommentTok{\# Peso Sináptico}
\FunctionTok{print}\NormalTok{(W)}
\end{Highlighting}
\end{Shaded}

\begin{verbatim}
## [1]  0.02  0.01 -0.02
\end{verbatim}

\begin{Shaded}
\begin{Highlighting}[]
\CommentTok{\# Erro total}
\FunctionTok{print}\NormalTok{(erro\_total)}
\end{Highlighting}
\end{Shaded}

\begin{verbatim}
##  [1] 3 3 3 3 3 3 3 3 3 3 3 3 3 3 3 3 3 3 3 3 3 4 4 4 4 4 4 4 4 3 3 3 3 3 3 3 3 3
## [39] 2 3 2 3 3 2 3 3 2 3 3 2 3 3 2 3 3 2 3 3 2 1 0 0 0 0 0 0 0 0 0 0 0 0 0 0 0 0
## [77] 0 0 0 0
\end{verbatim}

\begin{Shaded}
\begin{Highlighting}[]
\FunctionTok{plot}\NormalTok{(erro\_total,}\AttributeTok{type=}\StringTok{"l"}\NormalTok{)}
\end{Highlighting}
\end{Shaded}

\includegraphics{trabalho_rna_aula_2_files/figure-latex/unnamed-chunk-17-2.pdf}

\begin{Shaded}
\begin{Highlighting}[]
\CommentTok{\# Gráfico do Conjunto de Treinamento}
\FunctionTok{plot}\NormalTok{(X,}\AttributeTok{type=}\StringTok{"n"}\NormalTok{,}\AttributeTok{xlim=}\FunctionTok{c}\NormalTok{(}\SpecialCharTok{{-}}\FloatTok{1.5}\NormalTok{,}\FloatTok{1.5}\NormalTok{ ), }\AttributeTok{ylim=}\FunctionTok{c}\NormalTok{(}\SpecialCharTok{{-}}\FloatTok{1.5}\NormalTok{,}\FloatTok{1.5}\NormalTok{))}
\FunctionTok{points}\NormalTok{(x1[}\DecValTok{1}\NormalTok{],x1[}\DecValTok{2}\NormalTok{],}\AttributeTok{pch=}\FunctionTok{c}\NormalTok{(}\DecValTok{19}\NormalTok{),}\AttributeTok{cex=}\DecValTok{2}\NormalTok{,}\AttributeTok{col=}\StringTok{"2"}\NormalTok{)}
\FunctionTok{points}\NormalTok{(x2[}\DecValTok{1}\NormalTok{],x2[}\DecValTok{2}\NormalTok{],}\AttributeTok{pch=}\FunctionTok{c}\NormalTok{(}\DecValTok{19}\NormalTok{),}\AttributeTok{cex=}\DecValTok{2}\NormalTok{,}\AttributeTok{col=}\StringTok{"2"}\NormalTok{)}
\FunctionTok{points}\NormalTok{(x3[}\DecValTok{1}\NormalTok{],x3[}\DecValTok{2}\NormalTok{],}\AttributeTok{pch=}\FunctionTok{c}\NormalTok{(}\DecValTok{19}\NormalTok{),}\AttributeTok{cex=}\DecValTok{2}\NormalTok{,}\AttributeTok{col=}\StringTok{"2"}\NormalTok{)}
\FunctionTok{points}\NormalTok{(x4[}\DecValTok{1}\NormalTok{],x4[}\DecValTok{2}\NormalTok{],}\AttributeTok{pch=}\FunctionTok{c}\NormalTok{(}\DecValTok{15}\NormalTok{),}\AttributeTok{cex=}\DecValTok{2}\NormalTok{,}\AttributeTok{col=}\StringTok{"1"}\NormalTok{)}
\FunctionTok{legend}\NormalTok{(}\FloatTok{0.4}\NormalTok{,}\FloatTok{0.6}\NormalTok{,}\AttributeTok{legend=}\FunctionTok{c}\NormalTok{(}\StringTok{"0"}\NormalTok{,}\StringTok{"1"}\NormalTok{),}\AttributeTok{col=}\FunctionTok{c}\NormalTok{(}\DecValTok{2}\NormalTok{,}\DecValTok{1}\NormalTok{), }\AttributeTok{pch=}\FunctionTok{c}\NormalTok{(}\DecValTok{19}\NormalTok{,}\DecValTok{15}\NormalTok{))}

\NormalTok{x\_1}\OtherTok{\textless{}{-}} \SpecialCharTok{{-}}\NormalTok{(W[}\DecValTok{3}\NormalTok{]}\SpecialCharTok{/}\NormalTok{W[}\DecValTok{1}\NormalTok{])}

\NormalTok{x\_2}\OtherTok{\textless{}{-}} \SpecialCharTok{{-}}\NormalTok{(W[}\DecValTok{3}\NormalTok{]}\SpecialCharTok{/}\NormalTok{W[}\DecValTok{2}\NormalTok{])}

\FunctionTok{segments}\NormalTok{(x\_1,}\DecValTok{0}\NormalTok{,}\DecValTok{0}\NormalTok{,x\_2)}
\end{Highlighting}
\end{Shaded}

\includegraphics{trabalho_rna_aula_2_files/figure-latex/unnamed-chunk-17-3.pdf}

O Gráfico de erro continua estagnado, sem aprender. Neste caso
aumentou-se também as épocas para 80, a fim de gerar mais iterações e
permitir que o algoritimo aprenda até aproximar o erro ao zero.

\hypertarget{caso-2---dataset-iris}{%
\subsection{Caso 2 - dataset IRIS}\label{caso-2---dataset-iris}}

Neste caso podemos selecionar os valores de X o dataset fornecido pelas
dicas e os valores de saída y as classes\_bin e montar um número de
épocas máximo 20, pois o algoritimo aprende em até 9 com o set de eta
\textless- 0.01 e W \textless- c(0.1, 0, 1)

\begin{Shaded}
\begin{Highlighting}[]
\FunctionTok{library}\NormalTok{(}\StringTok{"plyr"}\NormalTok{)}
\FunctionTok{data}\NormalTok{(}\StringTok{"iris"}\NormalTok{)}
\NormalTok{dataset}\OtherTok{\textless{}{-}}\NormalTok{iris[}\FunctionTok{which}\NormalTok{(iris}\SpecialCharTok{$}\NormalTok{Species}\SpecialCharTok{!=}\StringTok{"versicolor"}\NormalTok{),][}\DecValTok{1}\SpecialCharTok{:}\DecValTok{2{-}3}\NormalTok{]}
\NormalTok{classes}\OtherTok{\textless{}{-}}\FunctionTok{as.numeric}\NormalTok{(dataset}\SpecialCharTok{$}\NormalTok{Species)}
\NormalTok{classes\_bin}\OtherTok{\textless{}{-}}\FunctionTok{mapvalues}\NormalTok{(classes,}\AttributeTok{from=}\FunctionTok{c}\NormalTok{(}\DecValTok{1}\NormalTok{,}\DecValTok{3}\NormalTok{),}\AttributeTok{to=}\FunctionTok{c}\NormalTok{(}\DecValTok{0}\NormalTok{,}\DecValTok{1}\NormalTok{))}
\FunctionTok{print}\NormalTok{((classes\_bin))}
\end{Highlighting}
\end{Shaded}

\begin{verbatim}
##   [1] 0 0 0 0 0 0 0 0 0 0 0 0 0 0 0 0 0 0 0 0 0 0 0 0 0 0 0 0 0 0 0 0 0 0 0 0 0
##  [38] 0 0 0 0 0 0 0 0 0 0 0 0 0 1 1 1 1 1 1 1 1 1 1 1 1 1 1 1 1 1 1 1 1 1 1 1 1
##  [75] 1 1 1 1 1 1 1 1 1 1 1 1 1 1 1 1 1 1 1 1 1 1 1 1 1 1
\end{verbatim}

\begin{Shaded}
\begin{Highlighting}[]
\NormalTok{x1}\OtherTok{\textless{}{-}}\NormalTok{dataset[,}\DecValTok{1}\NormalTok{]}\SpecialCharTok{/}\FunctionTok{max}\NormalTok{(dataset[,}\DecValTok{1}\NormalTok{])}
\NormalTok{x2}\OtherTok{\textless{}{-}}\NormalTok{dataset[,}\DecValTok{2}\NormalTok{]}\SpecialCharTok{/}\FunctionTok{max}\NormalTok{(dataset[,}\DecValTok{2}\NormalTok{])}

\NormalTok{X }\OtherTok{\textless{}{-}} \FunctionTok{cbind}\NormalTok{(x1, x2, }\FunctionTok{rep}\NormalTok{(}\DecValTok{1}\NormalTok{, }\FunctionTok{length}\NormalTok{(x1)))}
\NormalTok{Y }\OtherTok{\textless{}{-}}\NormalTok{ classes\_bin}

\NormalTok{W }\OtherTok{\textless{}{-}} \FunctionTok{c}\NormalTok{(}\FloatTok{0.1}\NormalTok{, }\DecValTok{0}\NormalTok{, }\DecValTok{1}\NormalTok{)}
\NormalTok{eta }\OtherTok{\textless{}{-}} \FloatTok{0.01}
\NormalTok{epoca\_max }\OtherTok{\textless{}{-}} \DecValTok{20}

\NormalTok{erro\_ite }\OtherTok{\textless{}{-}} \FunctionTok{rep}\NormalTok{(}\DecValTok{0}\NormalTok{, }\FunctionTok{nrow}\NormalTok{(X))}
\NormalTok{erro\_total }\OtherTok{\textless{}{-}} \FunctionTok{rep}\NormalTok{(}\DecValTok{0}\NormalTok{, epoca\_max)}

\ControlFlowTok{for}\NormalTok{ (epoca }\ControlFlowTok{in} \DecValTok{1}\SpecialCharTok{:}\NormalTok{epoca\_max) \{}
  \ControlFlowTok{for}\NormalTok{ (i }\ControlFlowTok{in} \DecValTok{1}\SpecialCharTok{:}\FunctionTok{nrow}\NormalTok{(X)) \{}
    \CommentTok{\# Calcular a saída da rede neural}
\NormalTok{    v }\OtherTok{\textless{}{-}} \FunctionTok{sum}\NormalTok{(X[i, ] }\SpecialCharTok{*}\NormalTok{ W)}
    \ControlFlowTok{if}\NormalTok{ (v }\SpecialCharTok{\textgreater{}} \DecValTok{0}\NormalTok{) \{}
\NormalTok{      y\_calc }\OtherTok{\textless{}{-}} \DecValTok{1}
\NormalTok{    \} }\ControlFlowTok{else}\NormalTok{ \{}
\NormalTok{      y\_calc }\OtherTok{\textless{}{-}} \DecValTok{0}
\NormalTok{    \}}
    \CommentTok{\# Calcular o erro}
\NormalTok{    erro }\OtherTok{\textless{}{-}}\NormalTok{ Y[i] }\SpecialCharTok{{-}}\NormalTok{ y\_calc}
    \CommentTok{\# Atualizar os pesos da rede neural}
\NormalTok{    delta }\OtherTok{\textless{}{-}}\NormalTok{ eta }\SpecialCharTok{*}\NormalTok{ erro }\SpecialCharTok{*}\NormalTok{ X[i, ]}
\NormalTok{    W }\OtherTok{\textless{}{-}}\NormalTok{ W }\SpecialCharTok{+}\NormalTok{ delta}
    \CommentTok{\# Armazenar o erro para a iteração atual}
\NormalTok{    erro\_ite[i] }\OtherTok{\textless{}{-}}\NormalTok{ erro }\SpecialCharTok{\^{}} \DecValTok{2}
\NormalTok{  \}}
\NormalTok{  erro\_total[epoca] }\OtherTok{\textless{}{-}} \FunctionTok{sum}\NormalTok{(erro\_ite)}
  \ControlFlowTok{if}\NormalTok{ (}\FunctionTok{sum}\NormalTok{(erro\_ite) }\SpecialCharTok{==} \DecValTok{0}\NormalTok{) \{}
    \ControlFlowTok{break}
\NormalTok{  \}}
\NormalTok{\}}

\FunctionTok{print}\NormalTok{(}\FunctionTok{paste}\NormalTok{(}\StringTok{"Número de épocas usadas no treinamento"}\NormalTok{, epoca, }\StringTok{"de um máximo de"}\NormalTok{, epoca\_max))}
\end{Highlighting}
\end{Shaded}

\begin{verbatim}
## [1] "Número de épocas usadas no treinamento 9 de um máximo de 20"
\end{verbatim}

\begin{Shaded}
\begin{Highlighting}[]
\FunctionTok{print}\NormalTok{(W)}
\end{Highlighting}
\end{Shaded}

\begin{verbatim}
##           x1           x2              
## -0.007101449  0.029600000 -0.010000000
\end{verbatim}

\begin{Shaded}
\begin{Highlighting}[]
\FunctionTok{print}\NormalTok{(erro\_total)}
\end{Highlighting}
\end{Shaded}

\begin{verbatim}
##  [1] 50 55 12  8  4  2  3  1  0  0  0  0  0  0  0  0  0  0  0  0
\end{verbatim}

\begin{Shaded}
\begin{Highlighting}[]
\FunctionTok{plot}\NormalTok{(erro\_total, }\AttributeTok{type =} \StringTok{"l"}\NormalTok{)}
\end{Highlighting}
\end{Shaded}

\includegraphics{trabalho_rna_aula_2_files/figure-latex/unnamed-chunk-18-1.pdf}

\begin{Shaded}
\begin{Highlighting}[]
\FunctionTok{plot}\NormalTok{(x1, x2, }\AttributeTok{pch =} \FunctionTok{ifelse}\NormalTok{(classes\_bin }\SpecialCharTok{==} \DecValTok{0}\NormalTok{, }\DecValTok{1}\NormalTok{, }\DecValTok{19}\NormalTok{), }\AttributeTok{col =}\NormalTok{ classes\_bin }\SpecialCharTok{+} \DecValTok{1}\NormalTok{)}
\FunctionTok{legend}\NormalTok{(}\StringTok{"topleft"}\NormalTok{, }\AttributeTok{legend =} \FunctionTok{c}\NormalTok{(}\StringTok{"Setosa"}\NormalTok{, }\StringTok{"Virginica"}\NormalTok{), }\AttributeTok{col =} \FunctionTok{c}\NormalTok{(}\DecValTok{1}\NormalTok{, }\DecValTok{2}\NormalTok{), }\AttributeTok{pch =} \FunctionTok{c}\NormalTok{(}\DecValTok{1}\NormalTok{, }\DecValTok{19}\NormalTok{))}
\NormalTok{x\_1 }\OtherTok{\textless{}{-}} \SpecialCharTok{{-}}\NormalTok{(W[}\DecValTok{3}\NormalTok{] }\SpecialCharTok{/}\NormalTok{ W[}\DecValTok{1}\NormalTok{])}
\NormalTok{x\_2 }\OtherTok{\textless{}{-}} \SpecialCharTok{{-}}\NormalTok{(W[}\DecValTok{3}\NormalTok{] }\SpecialCharTok{/}\NormalTok{ W[}\DecValTok{2}\NormalTok{])}
\FunctionTok{abline}\NormalTok{(}\AttributeTok{a =}\NormalTok{ x\_2, }\AttributeTok{b =} \SpecialCharTok{{-}}\NormalTok{W[}\DecValTok{1}\NormalTok{] }\SpecialCharTok{/}\NormalTok{ W[}\DecValTok{2}\NormalTok{])}
\end{Highlighting}
\end{Shaded}

\includegraphics{trabalho_rna_aula_2_files/figure-latex/unnamed-chunk-18-2.pdf}

No gráfico podemos ver que o algoritimo aprende e coloca a linha de
separação entre os sets de setosa e virginica, com o erro tendo a curva
de aprendizado exponencialmente acentuada à zero antes das cinco
iterações. Após as cinco primeiras iterações vemos que ele sobe o erro
para baixar tendendo a zero pós as 9 iteraçoes.

\end{document}
