% Options for packages loaded elsewhere
\PassOptionsToPackage{unicode}{hyperref}
\PassOptionsToPackage{hyphens}{url}
%
\documentclass[
]{article}
\usepackage{amsmath,amssymb}
\usepackage{lmodern}
\usepackage{iftex}
\ifPDFTeX
  \usepackage[T1]{fontenc}
  \usepackage[utf8]{inputenc}
  \usepackage{textcomp} % provide euro and other symbols
\else % if luatex or xetex
  \usepackage{unicode-math}
  \defaultfontfeatures{Scale=MatchLowercase}
  \defaultfontfeatures[\rmfamily]{Ligatures=TeX,Scale=1}
\fi
% Use upquote if available, for straight quotes in verbatim environments
\IfFileExists{upquote.sty}{\usepackage{upquote}}{}
\IfFileExists{microtype.sty}{% use microtype if available
  \usepackage[]{microtype}
  \UseMicrotypeSet[protrusion]{basicmath} % disable protrusion for tt fonts
}{}
\makeatletter
\@ifundefined{KOMAClassName}{% if non-KOMA class
  \IfFileExists{parskip.sty}{%
    \usepackage{parskip}
  }{% else
    \setlength{\parindent}{0pt}
    \setlength{\parskip}{6pt plus 2pt minus 1pt}}
}{% if KOMA class
  \KOMAoptions{parskip=half}}
\makeatother
\usepackage{xcolor}
\usepackage[margin=1in]{geometry}
\usepackage{color}
\usepackage{fancyvrb}
\newcommand{\VerbBar}{|}
\newcommand{\VERB}{\Verb[commandchars=\\\{\}]}
\DefineVerbatimEnvironment{Highlighting}{Verbatim}{commandchars=\\\{\}}
% Add ',fontsize=\small' for more characters per line
\usepackage{framed}
\definecolor{shadecolor}{RGB}{248,248,248}
\newenvironment{Shaded}{\begin{snugshade}}{\end{snugshade}}
\newcommand{\AlertTok}[1]{\textcolor[rgb]{0.94,0.16,0.16}{#1}}
\newcommand{\AnnotationTok}[1]{\textcolor[rgb]{0.56,0.35,0.01}{\textbf{\textit{#1}}}}
\newcommand{\AttributeTok}[1]{\textcolor[rgb]{0.77,0.63,0.00}{#1}}
\newcommand{\BaseNTok}[1]{\textcolor[rgb]{0.00,0.00,0.81}{#1}}
\newcommand{\BuiltInTok}[1]{#1}
\newcommand{\CharTok}[1]{\textcolor[rgb]{0.31,0.60,0.02}{#1}}
\newcommand{\CommentTok}[1]{\textcolor[rgb]{0.56,0.35,0.01}{\textit{#1}}}
\newcommand{\CommentVarTok}[1]{\textcolor[rgb]{0.56,0.35,0.01}{\textbf{\textit{#1}}}}
\newcommand{\ConstantTok}[1]{\textcolor[rgb]{0.00,0.00,0.00}{#1}}
\newcommand{\ControlFlowTok}[1]{\textcolor[rgb]{0.13,0.29,0.53}{\textbf{#1}}}
\newcommand{\DataTypeTok}[1]{\textcolor[rgb]{0.13,0.29,0.53}{#1}}
\newcommand{\DecValTok}[1]{\textcolor[rgb]{0.00,0.00,0.81}{#1}}
\newcommand{\DocumentationTok}[1]{\textcolor[rgb]{0.56,0.35,0.01}{\textbf{\textit{#1}}}}
\newcommand{\ErrorTok}[1]{\textcolor[rgb]{0.64,0.00,0.00}{\textbf{#1}}}
\newcommand{\ExtensionTok}[1]{#1}
\newcommand{\FloatTok}[1]{\textcolor[rgb]{0.00,0.00,0.81}{#1}}
\newcommand{\FunctionTok}[1]{\textcolor[rgb]{0.00,0.00,0.00}{#1}}
\newcommand{\ImportTok}[1]{#1}
\newcommand{\InformationTok}[1]{\textcolor[rgb]{0.56,0.35,0.01}{\textbf{\textit{#1}}}}
\newcommand{\KeywordTok}[1]{\textcolor[rgb]{0.13,0.29,0.53}{\textbf{#1}}}
\newcommand{\NormalTok}[1]{#1}
\newcommand{\OperatorTok}[1]{\textcolor[rgb]{0.81,0.36,0.00}{\textbf{#1}}}
\newcommand{\OtherTok}[1]{\textcolor[rgb]{0.56,0.35,0.01}{#1}}
\newcommand{\PreprocessorTok}[1]{\textcolor[rgb]{0.56,0.35,0.01}{\textit{#1}}}
\newcommand{\RegionMarkerTok}[1]{#1}
\newcommand{\SpecialCharTok}[1]{\textcolor[rgb]{0.00,0.00,0.00}{#1}}
\newcommand{\SpecialStringTok}[1]{\textcolor[rgb]{0.31,0.60,0.02}{#1}}
\newcommand{\StringTok}[1]{\textcolor[rgb]{0.31,0.60,0.02}{#1}}
\newcommand{\VariableTok}[1]{\textcolor[rgb]{0.00,0.00,0.00}{#1}}
\newcommand{\VerbatimStringTok}[1]{\textcolor[rgb]{0.31,0.60,0.02}{#1}}
\newcommand{\WarningTok}[1]{\textcolor[rgb]{0.56,0.35,0.01}{\textbf{\textit{#1}}}}
\usepackage{graphicx}
\makeatletter
\def\maxwidth{\ifdim\Gin@nat@width>\linewidth\linewidth\else\Gin@nat@width\fi}
\def\maxheight{\ifdim\Gin@nat@height>\textheight\textheight\else\Gin@nat@height\fi}
\makeatother
% Scale images if necessary, so that they will not overflow the page
% margins by default, and it is still possible to overwrite the defaults
% using explicit options in \includegraphics[width, height, ...]{}
\setkeys{Gin}{width=\maxwidth,height=\maxheight,keepaspectratio}
% Set default figure placement to htbp
\makeatletter
\def\fps@figure{htbp}
\makeatother
\setlength{\emergencystretch}{3em} % prevent overfull lines
\providecommand{\tightlist}{%
  \setlength{\itemsep}{0pt}\setlength{\parskip}{0pt}}
\setcounter{secnumdepth}{-\maxdimen} % remove section numbering
\ifLuaTeX
  \usepackage{selnolig}  % disable illegal ligatures
\fi
\IfFileExists{bookmark.sty}{\usepackage{bookmark}}{\usepackage{hyperref}}
\IfFileExists{xurl.sty}{\usepackage{xurl}}{} % add URL line breaks if available
\urlstyle{same} % disable monospaced font for URLs
\hypersetup{
  pdftitle={Trabalho RNA Continuo},
  pdfauthor={AfonsoBrandao},
  hidelinks,
  pdfcreator={LaTeX via pandoc}}

\title{Trabalho RNA Continuo}
\author{AfonsoBrandao}
\date{2023-03-07}

\begin{document}
\maketitle

\hypertarget{primeira-questuxe3o}{%
\subsection{Primeira questão}\label{primeira-questuxe3o}}

\begin{enumerate}
\def\labelenumi{\arabic{enumi})}
\tightlist
\item
  Aumente o número máximo de épocas (por exemplo, para 300) e apresente
  o gráfico de erro por época, como também o gráfico de sobreposição de
  saídas (desejada e estimada). Faça uma análise se há melhoras em
  relação aos valores anteriormente apresentados. Verifique o valor
  final de erro e veja se ele consegue chegar a 0 (zero).
\end{enumerate}

\begin{itemize}
\tightlist
\item
  Regressão modelo
\end{itemize}

\begin{Shaded}
\begin{Highlighting}[]
\CommentTok{\# Definindo o tamanho do vetor}
\NormalTok{N}\OtherTok{\textless{}{-}}\DecValTok{10}

\CommentTok{\# Criando o vetor X com valores de 1 a N}
\NormalTok{X}\OtherTok{\textless{}{-}}\FunctionTok{seq}\NormalTok{(}\DecValTok{1}\NormalTok{,N)}

\CommentTok{\# Criando o vetor Y com a função 2X + 3}
\NormalTok{Y}\OtherTok{\textless{}{-}}\DecValTok{2}\SpecialCharTok{*}\NormalTok{X}\SpecialCharTok{+}\DecValTok{3}

\CommentTok{\# Plotando os valores de X e Y em um gráfico de pontos}
\FunctionTok{plot}\NormalTok{(X,Y,}\AttributeTok{type=}\StringTok{"o"}\NormalTok{,}\AttributeTok{xlim=}\FunctionTok{c}\NormalTok{(}\DecValTok{0}\NormalTok{, }\DecValTok{10}\NormalTok{),}\AttributeTok{ylim=}\FunctionTok{c}\NormalTok{(}\DecValTok{0}\NormalTok{,}\DecValTok{25}\NormalTok{))}
\end{Highlighting}
\end{Shaded}

\includegraphics{trabalho_rna_continuo_files/figure-latex/unnamed-chunk-1-1.pdf}
- Ajustes

\begin{Shaded}
\begin{Highlighting}[]
\CommentTok{\# Normalizando os vetores entre 0 e 1 e transpondo}
\NormalTok{X}\OtherTok{\textless{}{-}}\NormalTok{X}\SpecialCharTok{/}\FunctionTok{max}\NormalTok{(X)}
\NormalTok{Y}\OtherTok{\textless{}{-}}\NormalTok{Y}\SpecialCharTok{/}\FunctionTok{max}\NormalTok{(Y)}
\NormalTok{X}\OtherTok{\textless{}{-}}\FunctionTok{t}\NormalTok{(}\FunctionTok{t}\NormalTok{(X))}

\CommentTok{\# Adicionando um vetor de bias à matriz X}
\NormalTok{bias}\OtherTok{\textless{}{-}}\DecValTok{1}
\NormalTok{X}\OtherTok{\textless{}{-}}\FunctionTok{cbind}\NormalTok{(X,bias)}

\CommentTok{\# Imprimindo as matrizes X e Y}
\FunctionTok{print}\NormalTok{(X)}
\end{Highlighting}
\end{Shaded}

\begin{verbatim}
##           bias
##  [1,] 0.1    1
##  [2,] 0.2    1
##  [3,] 0.3    1
##  [4,] 0.4    1
##  [5,] 0.5    1
##  [6,] 0.6    1
##  [7,] 0.7    1
##  [8,] 0.8    1
##  [9,] 0.9    1
## [10,] 1.0    1
\end{verbatim}

\begin{Shaded}
\begin{Highlighting}[]
\FunctionTok{print}\NormalTok{(}\FunctionTok{cbind}\NormalTok{(Y))}
\end{Highlighting}
\end{Shaded}

\begin{verbatim}
##               Y
##  [1,] 0.2173913
##  [2,] 0.3043478
##  [3,] 0.3913043
##  [4,] 0.4782609
##  [5,] 0.5652174
##  [6,] 0.6521739
##  [7,] 0.7391304
##  [8,] 0.8260870
##  [9,] 0.9130435
## [10,] 1.0000000
\end{verbatim}

\begin{itemize}
\tightlist
\item
  Definindo variáveis e treinando
\end{itemize}

\begin{Shaded}
\begin{Highlighting}[]
\CommentTok{\# Definindo o número máximo de épocas e a taxa de aprendizado}
\NormalTok{epoca\_max}\OtherTok{\textless{}{-}}\DecValTok{300}
\NormalTok{eta}\OtherTok{\textless{}{-}}\FloatTok{0.01}

\CommentTok{\# Inicializando o vetor de pesos W}
\NormalTok{W}\OtherTok{\textless{}{-}}\FunctionTok{c}\NormalTok{(}\DecValTok{0}\NormalTok{,}\DecValTok{1}\NormalTok{)}

\CommentTok{\# Inicializando vetores para armazenar os erros das iterações e das épocas}
\NormalTok{err\_iter}\OtherTok{\textless{}{-}}\FunctionTok{rep}\NormalTok{(}\DecValTok{0}\NormalTok{,N)}
\NormalTok{err\_epoc}\OtherTok{\textless{}{-}}\FunctionTok{rep}\NormalTok{(}\DecValTok{0}\NormalTok{,epoca\_max)}

\CommentTok{\# Iniciando o loop de treinamento}
\ControlFlowTok{for}\NormalTok{ (epoca }\ControlFlowTok{in} \DecValTok{1}\SpecialCharTok{:}\NormalTok{epoca\_max) \{}
  \ControlFlowTok{for}\NormalTok{ (i }\ControlFlowTok{in} \DecValTok{1}\SpecialCharTok{:}\NormalTok{N) \{}
    \CommentTok{\# Calculando a saída estimada para a entrada atual}
\NormalTok{    v }\OtherTok{\textless{}{-}} \FunctionTok{sum}\NormalTok{(X[i,]}\SpecialCharTok{*}\NormalTok{W)}
    \CommentTok{\# Calculando o erro entre a saída desejada e a saída estimada}
\NormalTok{    erro }\OtherTok{\textless{}{-}}\NormalTok{ Y[i] }\SpecialCharTok{{-}}\NormalTok{ v}
    \CommentTok{\# Calculando o delta para atualizar os pesos}
\NormalTok{    delta }\OtherTok{\textless{}{-}}\NormalTok{ eta}\SpecialCharTok{*}\NormalTok{erro}\SpecialCharTok{*}\NormalTok{X[i,]}
    \CommentTok{\# Atualizando os pesos}
\NormalTok{    W }\OtherTok{\textless{}{-}}\NormalTok{ W }\SpecialCharTok{+}\NormalTok{ delta}
    \CommentTok{\# Armazenando o erro da iteração atual}
\NormalTok{    err\_iter[i] }\OtherTok{\textless{}{-}} \FloatTok{0.5}\SpecialCharTok{*}\NormalTok{(erro}\SpecialCharTok{\^{}}\DecValTok{2}\NormalTok{)}
\NormalTok{  \}}
  \CommentTok{\# Armazenando o erro da época atual}
\NormalTok{  err\_epoc[epoca] }\OtherTok{\textless{}{-}} \FunctionTok{sum}\NormalTok{(err\_iter)}
\NormalTok{\}}

\CommentTok{\# Plotando o erro quadrático médio ao longo das épocas}
\FunctionTok{plot}\NormalTok{(err\_epoc, }\AttributeTok{type =} \StringTok{"line"}\NormalTok{, }\AttributeTok{xlab =} \StringTok{"época"}\NormalTok{, }\AttributeTok{ylab =} \StringTok{"erro quadrático médio"}\NormalTok{)}
\end{Highlighting}
\end{Shaded}

\begin{verbatim}
## Warning in plot.xy(xy, type, ...): plot type 'line' will be truncated to first
## character
\end{verbatim}

\includegraphics{trabalho_rna_continuo_files/figure-latex/unnamed-chunk-3-1.pdf}
- Pesos finais

\begin{Shaded}
\begin{Highlighting}[]
\FunctionTok{print}\NormalTok{(W)}
\end{Highlighting}
\end{Shaded}

\begin{verbatim}
##                bias 
## 0.7146150 0.2231222
\end{verbatim}

\begin{itemize}
\tightlist
\item
  Calculando a saída estimada para todas as entradas e Plotando as
  saídas desejadas e estimadas em um gráfico
\end{itemize}

\begin{Shaded}
\begin{Highlighting}[]
\NormalTok{Y\_estimado}\OtherTok{\textless{}{-}}\NormalTok{X[,}\DecValTok{1}\NormalTok{]}\SpecialCharTok{*}\NormalTok{W[}\DecValTok{1}\NormalTok{]}\SpecialCharTok{+}\NormalTok{W[}\DecValTok{2}\NormalTok{]}

\FunctionTok{plot}\NormalTok{(X[,}\DecValTok{1}\NormalTok{],Y,}\AttributeTok{type =}\StringTok{"n"}\NormalTok{)}
\FunctionTok{points}\NormalTok{(X[,}\DecValTok{1}\NormalTok{],Y,}\AttributeTok{pch=}\DecValTok{1}\NormalTok{,}\AttributeTok{type=}\StringTok{"o"}\NormalTok{,}\AttributeTok{col=}\StringTok{"1"}\NormalTok{)}
\FunctionTok{points}\NormalTok{(X[,}\DecValTok{1}\NormalTok{],Y\_estimado,}\AttributeTok{pch=}\DecValTok{16}\NormalTok{,}\AttributeTok{type=}\StringTok{"p"}\NormalTok{,}\AttributeTok{col=}\StringTok{"2"}\NormalTok{)}
\FunctionTok{legend}\NormalTok{(}\FloatTok{0.1}\NormalTok{,}\FloatTok{0.9}\NormalTok{,}\AttributeTok{legend=}\FunctionTok{c}\NormalTok{(}\StringTok{"Y\_desejado"}\NormalTok{,}\StringTok{"Y\_estimado"}\NormalTok{),}\AttributeTok{col=}\FunctionTok{c}\NormalTok{(}\DecValTok{1}\NormalTok{,}\DecValTok{2}\NormalTok{),}\AttributeTok{pch=}\FunctionTok{c}\NormalTok{(}\DecValTok{1}\NormalTok{,}\DecValTok{16}\NormalTok{))}
\end{Highlighting}
\end{Shaded}

\includegraphics{trabalho_rna_continuo_files/figure-latex/unnamed-chunk-5-1.pdf}

\begin{Shaded}
\begin{Highlighting}[]
\FunctionTok{print}\NormalTok{(}\FunctionTok{paste}\NormalTok{(}\StringTok{"Erro final do treinamento"}\NormalTok{, err\_epoc[epoca\_max]))}
\end{Highlighting}
\end{Shaded}

\begin{verbatim}
## [1] "Erro final do treinamento 0.0103700261799258"
\end{verbatim}

\hypertarget{resposta}{%
\subsubsection{Resposta}\label{resposta}}

Considerando o contexto abordado, foi possível observar um resultado
significativo ao aumentar o número máximo de épocas para 300 no
treinamento da rede neural. A partir do gráfico de erro por época,
verificou-se que o erro inicialmente elevado apresentou uma rápida
redução nas primeiras iterações, mas, posteriormente, a redução foi
gradual, culminando em uma estabilização em torno de 0,103. Esse
comportamento sugere que a rede neural está sendo capaz de aprender a
relação entre as entradas e as saídas, porém ainda é possível considerar
melhorias no processo.

Com relação ao gráfico de sobreposição de saídas, foi possível constatar
que as saídas estimadas apresentaram uma boa aproximação em relação às
saídas desejadas, com discrepâncias mínimas entre elas. Esse resultado
evidencia que a rede neural está realizando uma aproximação satisfatória
da função que gera as saídas, contribuindo para aprimoramentos futuros
no processo de treinamento.

\hypertarget{segunda-questuxe3o}{%
\subsection{Segunda Questão}\label{segunda-questuxe3o}}

\begin{enumerate}
\def\labelenumi{\arabic{enumi})}
\setcounter{enumi}{1}
\tightlist
\item
  Agora, com este novo valor de época, altere o valor do eta para, por
  exemplo, 0.1. Análise o gráfico de erro e verifique se o ponto de
  diminuição abrupta (ponto em que faz o formato de ``cotovelo'' do
  gráfico) é anterior ou posterior ao do experimento anterior. Depois,
  refaça as tarefas pedidas no exercício 1 e compare os desempenhos. Por
  fim, conclua em que implica a alteração deste parâmetro e faça uma
  interpretação do que se espera acontecer se o valor deste parâmetro
  por aumentado (tendendo a 1).
\end{enumerate}

\begin{itemize}
\tightlist
\item
  Segue código condensado mudando a eta\textless- 0.01
\end{itemize}

\begin{Shaded}
\begin{Highlighting}[]
\FunctionTok{print}\NormalTok{(}\FunctionTok{paste}\NormalTok{(}\StringTok{"Erro final do treinamento"}\NormalTok{, err\_epoc[epoca\_max]))}
\end{Highlighting}
\end{Shaded}

\begin{verbatim}
## [1] "Erro final do treinamento 0.0103700261799258"
\end{verbatim}

\end{document}
